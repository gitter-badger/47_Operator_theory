\documentclass[12pt]{article}
\usepackage{pmmeta}
\pmcanonicalname{MultiplicationOperatorOnL2}
\pmcreated{2013-03-22 15:42:28}
\pmmodified{2013-03-22 15:42:28}
\pmowner{scineram}{4030}
\pmmodifier{scineram}{4030}
\pmtitle{multiplication operator on $L^2$}
\pmrecord{8}{37654}
\pmprivacy{1}
\pmauthor{scineram}{4030}
\pmtype{Definition}
\pmcomment{trigger rebuild}
\pmclassification{msc}{47B38}
%\pmkeywords{multiplication operator}
\pmrelated{operator}
\pmdefines{multiplication operator}

\endmetadata

% this is the default PlanetMath preamble.  as your knowledge
% of TeX increases, you will probably want to edit this, but
% it should be fine as is for beginners.

% almost certainly you want these
\usepackage{amssymb}
\usepackage{amsmath}
\usepackage{amsfonts}
\usepackage{amsthm}
% used for TeXing text within eps files
%\usepackage{psfrag}
% need this for including graphics (\includegraphics)
%\usepackage{graphicx}
% for neatly defining theorems and propositions
\usepackage{amsthm}
% making logically defined graphics
%%%\usepackage{xypic}

% there are many more packages, add them here as you need them

% define commands here
\begin{document}
Let $(X,\mathcal{A},\mu)$ be a measure space and $f \colon X \to \mathbb{K}$ a measurable function. Then $M_f \colon \phi \mapsto f \phi$ is the multiplication operator with $f$ defined on the subspace $Dom(M_f)=\{\phi \in L^2_\mathbb{K}(X,\mathcal{A},\mu) \colon f \phi \in L^2_\mathbb{K}(X,\mathcal{A},\mu)\}$. It plays an important role in quantum mechanics where the multiplication with the coordinates on $\mathbb{R}^n$ is the position operator.
%%%%%
%%%%%
\end{document}
