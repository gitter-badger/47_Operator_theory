\documentclass[12pt]{article}
\usepackage{pmmeta}
\pmcanonicalname{InvariantSubspaceProblem}
\pmcreated{2013-03-22 17:24:02}
\pmmodified{2013-03-22 17:24:02}
\pmowner{asteroid}{17536}
\pmmodifier{asteroid}{17536}
\pmtitle{invariant subspace problem}
\pmrecord{5}{39771}
\pmprivacy{1}
\pmauthor{asteroid}{17536}
\pmtype{Conjecture}
\pmcomment{trigger rebuild}
\pmclassification{msc}{47A15}
\pmclassification{msc}{46-00}
\pmsynonym{invariant subspace conjecture}{InvariantSubspaceProblem}

% this is the default PlanetMath preamble.  as your knowledge
% of TeX increases, you will probably want to edit this, but
% it should be fine as is for beginners.

% almost certainly you want these
\usepackage{amssymb}
\usepackage{amsmath}
\usepackage{amsfonts}

% used for TeXing text within eps files
%\usepackage{psfrag}
% need this for including graphics (\includegraphics)
%\usepackage{graphicx}
% for neatly defining theorems and propositions
%\usepackage{amsthm}
% making logically defined graphics
%%%\usepackage{xypic}

% there are many more packages, add them here as you need them

% define commands here

\begin{document}
Initially formulated for Banach spaces, the {\bf invariant subspace conjecture} stated the following:

\emph{Let $X$ be a complex Banach space. Then every bounded operator $T$ in $X$ has a non-trivial \PMlinkname{closed}{ClosedSet} invariant subspace, i.e. there exists a closed vector subspace $S \subset X$ such that $S \neq 0$, $S \neq X$ and $T(S) \subseteq S$.}

This conjecture was proven to be false when P. Enflo (1975) and \PMlinkescapetext{C}. Read (1984) gave examples of bounded operators which did not have the above property.

However, if one considers only Hilbert spaces, this is still an open problem. Today the {\bf invariant subspace conjecture}  is formulated as follows:

\emph{Let $H$ be a complex Hilbert space. Then every bounded operator $T$ in $H$ has a non-trivial \PMlinkescapetext{closed} invariant subspace, i.e. there exists a closed vector subspace $S \subset H$ such that $S \neq 0$, $S \neq H$ and $T(S) \subseteq S$.}
%%%%%
%%%%%
\end{document}
