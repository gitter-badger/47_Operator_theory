\documentclass[12pt]{article}
\usepackage{pmmeta}
\pmcanonicalname{KatoRellichTheorem}
\pmcreated{2013-03-22 14:52:59}
\pmmodified{2013-03-22 14:52:59}
\pmowner{Koro}{127}
\pmmodifier{Koro}{127}
\pmtitle{Kato-Rellich theorem}
\pmrecord{7}{36562}
\pmprivacy{1}
\pmauthor{Koro}{127}
\pmtype{Theorem}
\pmcomment{trigger rebuild}
\pmclassification{msc}{47A55}
\pmsynonym{Rellich-Kato theorem}{KatoRellichTheorem}
\pmdefines{A-bounded}
\pmdefines{A-bound}

\endmetadata

% this is the default PlanetMath preamble.  as your knowledge
% of TeX increases, you will probably want to edit this, but
% it should be fine as is for beginners.

% almost certainly you want these
\usepackage{amssymb}
\usepackage{amsmath}
\usepackage{amsfonts}
\usepackage{mathrsfs}

% used for TeXing text within eps files
%\usepackage{psfrag}
% need this for including graphics (\includegraphics)
%\usepackage{graphicx}
% for neatly defining theorems and propositions
\usepackage{amsthm}
% making logically defined graphics
%%%\usepackage{xypic}

% there are many more packages, add them here as you need them

% define commands here
\newtheorem{theorem*}{Theorem}
\newcommand{\C}{\mathbb{C}}
\newcommand{\R}{\mathbb{R}}
\newcommand{\N}{\mathbb{N}}
\newcommand{\Z}{\mathbb{Z}}
\newcommand{\Per}{\operatorname{Per}}
\begin{document}
Let $\mathcal{H}$ be a Hilbert space, $A\colon D(A)\subset \mathcal{H}\to
\mathcal{H}$ a self-adjoint operator and $B\colon D(B)\subset\mathcal{H}\to
\mathcal{H}$ a symmetric operator with $D(A)\subset D(B)$.

We say that $B$ is $A$-bounded if there are positive constants
$\alpha,\beta$ such that $$\|Bx\|\leq \alpha\|Ax\|+\beta\|x\|$$ for all
$x\in D(A)$, and we say that $\alpha$ is an \emph{$A$-bound} for $B$.

\begin{theorem*}(Kato-Rellich) If $B$ is \emph{$A$-bounded} with $A$-bound smaller
than $1$, then $A+B$ is self-adjoint on $D(A)$, and essentially
self-adjoint on any core of $A$. Moreover, if $A$ is bounded below, then
so is $A+B$.
\end{theorem*}
%%%%%
%%%%%
\end{document}
