\documentclass[12pt]{article}
\usepackage{pmmeta}
\pmcanonicalname{BaerRing}
\pmcreated{2013-03-22 15:52:07}
\pmmodified{2013-03-22 15:52:07}
\pmowner{CWoo}{3771}
\pmmodifier{CWoo}{3771}
\pmtitle{Baer ring}
\pmrecord{6}{37864}
\pmprivacy{1}
\pmauthor{CWoo}{3771}
\pmtype{Definition}
\pmcomment{trigger rebuild}
\pmclassification{msc}{47C10}
\pmclassification{msc}{47C15}
\pmclassification{msc}{16U99}
\pmdefines{Baer *-ring}
\pmdefines{Rickart ring}

\usepackage{amssymb,amscd}
\usepackage{amsmath}
\usepackage{amsfonts}

% used for TeXing text within eps files
%\usepackage{psfrag}
% need this for including graphics (\includegraphics)
%\usepackage{graphicx}
% for neatly defining theorems and propositions
%\usepackage{amsthm}
% making logically defined graphics
%%%\usepackage{xypic}

% define commands here
\begin{document}
\subsubsection*{Baer Rings}

Let $R$ be a ring with multiplicative identity $1$.  Then $R$ is called a left \emph{Baer ring} if, for any subset $S$ of $R$, the left annihilator of $S$ is left principal, generated by an idempotent:
$$\operatorname{l.ann}(S):=\lbrace r\in R\mid rS=0\rbrace=Re.$$

A right Baer ring is defined similarly, by replacing the word left with right above.  It turns out that a left Baer ring is a right Baer ring, and vice versa, so we may drop the word left or right in the name.

Clearly, a domain is a Baer ring.  And, by Wedderburn-Artin theorem, a semisimple ring is also a Baer ring.  Another example, found in operator theory, is the ring of bounded linear operators on a Hilbert space.

A closely related concept is that of a left (right) \emph{Baer *-ring}: it is a ring with involution $*$ such that the left (right) annihilator of any subset is left (right) principal, generated by a projection (an idempotent that is in addition a self-adjoint element).

A left Baer *-ring is a left Baer ring, a right Baer *-ring is a right Baer ring. And, interestingly, the notion of left and right is also redundent for Baer *-rings.  For example, let's show left means right.  Since a left Baer * ring $R$ is left, and consequently right Baer, the right annihilator of a subset $S$ has the form $\operatorname{r.ann}(S)=eR$ for some idempotent $e\in R$.  Since $Se=0$, $e^*S^*=0$ or that $\operatorname{l.ann}(S^*)=Re^*$.  But $R$ is left Baer *, $\operatorname{l.ann}(S^*)=Rf$ for some projection $f\in R$.  So $fR=f^*R^*=(Rf)^*=(Re^*)^*=e^{**}R^*=eR=\operatorname{r.ann}(S)$.

\subsubsection*{Rickart Rings}

A closely related type of rings is called a left \emph{Rickart ring}.  A ring is left Rickart if the left annihilator of any element $a\in R$ is left principal, generated by an idempotent:
$$l.ann(a):=\lbrace r\in R\mid ra=0\rbrace = Re$$
where $e=ee$.  A right Rickart ring is similarly defined.  Clearly, a Baer ring is both right and left Rickart.  However, the converse is not always true.  In fact, right Rickart and left Rickart are not the same.  An example of a left Rickart ring that is not right Rickart can be found in the second reference below.  It may be shown that if all idempotents are central (lying in the center) in a left Rickart ring $R$, then $R$ is also right Rickart (and vice versa).

\begin{thebibliography}{9}
\bibitem{edwards} I. Kaplansky, \emph{Rings of Operators}, W. A. Benjamin, Inc., New York, 1968.
\bibitem{tylam} T. Y. Lam, \emph{Lectures on Modules and Rings}, Springer, New York, 1998.
\end{thebibliography}
%%%%%
%%%%%
\end{document}
