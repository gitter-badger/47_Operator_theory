\documentclass[12pt]{article}
\usepackage{pmmeta}
\pmcanonicalname{ErgodicityOfAMapInTermsOfItsInducedOperator}
\pmcreated{2013-03-22 17:59:22}
\pmmodified{2013-03-22 17:59:22}
\pmowner{asteroid}{17536}
\pmmodifier{asteroid}{17536}
\pmtitle{ergodicity of a map in terms of its induced operator}
\pmrecord{6}{40500}
\pmprivacy{1}
\pmauthor{asteroid}{17536}
\pmtype{Theorem}
\pmcomment{trigger rebuild}
\pmclassification{msc}{47A35}
\pmclassification{msc}{37A30}
\pmclassification{msc}{37A25}
\pmclassification{msc}{28D05}

% this is the default PlanetMath preamble.  as your knowledge
% of TeX increases, you will probably want to edit this, but
% it should be fine as is for beginners.

% almost certainly you want these
\usepackage{amssymb}
\usepackage{amsmath}
\usepackage{amsfonts}

% used for TeXing text within eps files
%\usepackage{psfrag}
% need this for including graphics (\includegraphics)
%\usepackage{graphicx}
% for neatly defining theorems and propositions
%\usepackage{amsthm}
% making logically defined graphics
%%%\usepackage{xypic}

% there are many more packages, add them here as you need them

% define commands here

\begin{document}
{\bf Theorem -} Let $(X, \mathfrak{B}, \mu)$ be a probability space and $T:X \longrightarrow X$ a measure-preserving transformation. The following statements are equivalent:

\begin{enumerate}
\item - $T$ is ergodic.
\item - If $f$ is a measurable function and $f\circ T =f$ \PMlinkname{a.e.}{AlmostSurely}, then $f$ is constant a.e.
\item - If $f$ is a measurable function and $f\circ T \geq f$ a.e., then $f$ is constant a.e.
\item - If $f \in L^2(X)$ and $f\circ T =f$ a.e., then $f$ is constant a.e..
\item - If $f \in L^p(X)$, with $p \geq 1$, and $f\circ T =f$ a.e., then $f$ is constant a.e.
\end{enumerate}

$\,$

Let $U_T$ denote the \PMlinkname{operator induced by $T$}{OperatorInducedByAMeasurePreservingMap}, i.e. the operator defined by $U_T f:= f \circ T$. The statements above are statements about $U_T$. The above theorem can be rewritten as follows:

$\,$

{\bf Theorem -} Let $(X, \mathfrak{B}, \mu)$ be a probability space and $T:X \longrightarrow X$ a measure-preserving transformation. The following statements are equivalent:

\begin{enumerate}
\item - $T$ is ergodic.
\item - The only fixed points of $U_T$ are the functions that are constant a.e.
\item - If $f$ a measurable function and $U_T f \geq f$ a.e., then $f$ is constant a.e.
\item - The eigenspace of $U_T$ (seen as an operator in $L^p(X)$, with $p \geq 1$) associated with the eigenvalue $1$, is one-dimensional and consists of functions that are constant a.e.
\end{enumerate}
%%%%%
%%%%%
\end{document}
