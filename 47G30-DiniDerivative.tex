\documentclass[12pt]{article}
\usepackage{pmmeta}
\pmcanonicalname{DiniDerivative}
\pmcreated{2013-03-22 13:57:00}
\pmmodified{2013-03-22 13:57:00}
\pmowner{lha}{3057}
\pmmodifier{lha}{3057}
\pmtitle{Dini derivative}
\pmrecord{11}{34714}
\pmprivacy{1}
\pmauthor{lha}{3057}
\pmtype{Definition}
\pmcomment{trigger rebuild}
\pmclassification{msc}{47G30}

\endmetadata

% this is the default PlanetMath preamble.  as your knowledge
% of TeX increases, you will probably want to edit this, but
% it should be fine as is for beginners.

% almost certainly you want these
\usepackage{amssymb}
\usepackage{amsmath}
\usepackage{amsfonts}

% used for TeXing text within eps files
%\usepackage{psfrag}
% need this for including graphics (\includegraphics)
%\usepackage{graphicx}
% for neatly defining theorems and propositions
%\usepackage{amsthm}
% making logically defined graphics
%%%\usepackage{xypic}

% there are many more packages, add them here as you need them

% define commands here
\begin{document}
The {\bf upper Dini derivative} of a continuous function, $f:{\bf R} \mapsto {\bf R}$, denoted by $f'_+$, is defined as
$$
    f'_+(t) = \lim_{h\rightarrow 0^+} \sup \frac{f(t + h) - f(t)}{h}.
$$
The {\bf lower Dini derivative}, $f'_-$, is defined as
$$
    f'_-(t) = \lim_{h\rightarrow 0^+} \inf \frac{f(t + h) - f(t)}{h}.
$$

Remark: Sometimes the notation $D^+ f(t)$ is used instead of $f'_+(t)$, and $D^- f(t)$ is used instead of $f'_-(t)$.

Remark: Like conventional derivatives, Dini derivatives do not always exist.

If $f$ is defined on a vector space, then the upper Dini derivative at $t$ in the direction $d$ is denoted
$$
    f'_+ (t,d) = \lim_{h\rightarrow 0^+} \sup \frac{f(t + hd) - f(t)}{h}.
$$

If $f$ is locally Lipschitz then $D^+ f$ is finite.  If $f$ is differentiable at $t$ then the Dini derivative at $t$ is the derivative at $t$.
%%%%%
%%%%%
\end{document}
