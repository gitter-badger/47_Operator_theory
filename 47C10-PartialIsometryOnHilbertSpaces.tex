\documentclass[12pt]{article}
\usepackage{pmmeta}
\pmcanonicalname{PartialIsometryOnHilbertSpaces}
\pmcreated{2013-03-22 18:35:00}
\pmmodified{2013-03-22 18:35:00}
\pmowner{karstenb}{16623}
\pmmodifier{karstenb}{16623}
\pmtitle{partial isometry on Hilbert spaces}
\pmrecord{7}{41309}
\pmprivacy{1}
\pmauthor{karstenb}{16623}
\pmtype{Definition}
\pmcomment{trigger rebuild}
\pmclassification{msc}{47C10}

\usepackage{amssymb}
\usepackage{amsmath}
\usepackage{amsfonts}
\usepackage{amsthm}
\usepackage{mathrsfs}
\usepackage[sort&compress]{natbib}

%\usepackage{psfrag}
%\usepackage{graphicx}
%%%\usepackage{xypic}

%theorems
\theoremstyle{definition}
\newtheorem{Def}{Definition}

\theoremstyle{plain}
\newtheorem{Lem}{Theorem}
\newtheorem{Lem2}{Lemma}
\newtheorem{Cor}{Corollary}
\newtheorem{Rem}{Remark}
\newtheorem{Prop}{Proposition}


% commands
%\newcommand{\comment}[1]{\small{(\,\textit{#1}\;)}}
\newcommand{\scal}[2]{\langle #1, #2 \rangle}
\newcommand{\ran}{\mathrm{ran}}


\begin{document}
\begin{Def}
Let $\mathscr{H}$ and $\mathscr{K}$ be Hilbert spaces. An operator $W \in L(\mathscr{H}, \mathscr{K})$ is called a \emph{partial isometry} if $W$ is an isometry on $M = (\ker W)^{\perp}$. We then call $M = (\ker W)^{\perp}$ the \emph{initial space} and $N = WM$ \emph{final space} of $W$.
\end{Def}

We need to show that the above definition is compatible with the general definition of partial isometry on rings. Indeed we have the following:

\begin{Prop}
$W \in L(\mathscr{H},\mathscr{K})$ is a partial isometry iff $W^* W$ is a projection from $\mathscr{H}$ to $M$.
\end{Prop}

\begin{proof}
We have:
\begin{align*}
W & \ \text{partial isometry with initial space} \ M \\
\Leftrightarrow \scal{Wf}{Wg} &= \scal{f}{g} \ \forall \ f,g \in M \\
\Leftrightarrow \scal{W^* W f}{g} &= \scal{f}{g} \ \forall \ f \in M, g \in \mathscr{H} \\
\Leftrightarrow W^* W f &= f, f \in M \\
\text{and} \ W^* W f &= 0, f \in M^{\perp} = \ker W
\end{align*}
\end{proof}

\begin{Rem}
If $W \in L(\mathscr{H}, \mathscr{K})$ is a partial isometry with initial space $M$ and final space $N$ we have:
\begin{align*}
W^* (Wf) &= f \ \forall \ f \in M \\
\ker W^* &= (\ran W)^{\perp} = N^{\perp}
\end{align*}

Thus $N$ is the initial space and $M$ the final space of $W^*$.
\end{Rem}


%%%%%
%%%%%
\end{document}
