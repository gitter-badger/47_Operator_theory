\documentclass[12pt]{article}
\usepackage{pmmeta}
\pmcanonicalname{EigenvaluesOfNormalOperators}
\pmcreated{2013-03-22 17:33:32}
\pmmodified{2013-03-22 17:33:32}
\pmowner{asteroid}{17536}
\pmmodifier{asteroid}{17536}
\pmtitle{eigenvalues of normal operators}
\pmrecord{10}{39965}
\pmprivacy{1}
\pmauthor{asteroid}{17536}
\pmtype{Theorem}
\pmcomment{trigger rebuild}
\pmclassification{msc}{47B15}
\pmclassification{msc}{47A75}
\pmclassification{msc}{47A15}
\pmclassification{msc}{47A10}
\pmclassification{msc}{15A18}

% this is the default PlanetMath preamble.  as your knowledge
% of TeX increases, you will probably want to edit this, but
% it should be fine as is for beginners.

% almost certainly you want these
\usepackage{amssymb}
\usepackage{amsmath}
\usepackage{amsfonts}

% used for TeXing text within eps files
%\usepackage{psfrag}
% need this for including graphics (\includegraphics)
%\usepackage{graphicx}
% for neatly defining theorems and propositions
%\usepackage{amsthm}
% making logically defined graphics
%%%\usepackage{xypic}

% there are many more packages, add them here as you need them

% define commands here

\begin{document}
Let $H$ be a Hilbert space and $B(H)$ the algebra of bounded operators in $H$. Suppose $T \in B(H)$ is a normal operator. Then

\begin{enumerate}
\item - If $\lambda \in \mathbb{C}$ is an eigenvalue of $T$, then $\overline{\lambda}$ is an eigenvalue of $T^*$ (the adjoint operator of $T$) for the same eigenvector.

\item - Eigenvectors of $T$ associated with distinct eigenvalues are orthogonal.

\end{enumerate}


{\bf Remark -} It is known that for any linear operator eigenvectors associated with distinct eigenvalues are linearly independent. \PMlinkescapetext{Point} 2 strengthens this result for normal operators.
%%%%%
%%%%%
\end{document}
