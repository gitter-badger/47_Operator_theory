\documentclass[12pt]{article}
\usepackage{pmmeta}
\pmcanonicalname{BakerCampbellHausdorffFormulae}
\pmcreated{2013-03-22 13:39:51}
\pmmodified{2013-03-22 13:39:51}
\pmowner{Mathprof}{13753}
\pmmodifier{Mathprof}{13753}
\pmtitle{Baker-Campbell-Hausdorff formula(e)}
\pmrecord{13}{34321}
\pmprivacy{1}
\pmauthor{Mathprof}{13753}
\pmtype{Definition}
\pmcomment{trigger rebuild}
\pmclassification{msc}{47A05}
\pmsynonym{BCH formula}{BakerCampbellHausdorffFormulae}
\pmsynonym{Baker-Campbell-Hausdorff formula}{BakerCampbellHausdorffFormulae}
\pmsynonym{Baker-Campbell-Hausdorff formulae}{BakerCampbellHausdorffFormulae}

\endmetadata

% this is the default PlanetMath preamble.  as your knowledge
% of TeX increases, you will probably want to edit this, but
% it should be fine as is for beginners.

% almost certainly you want these
\usepackage{amssymb}
\usepackage{amsmath}
\usepackage{amsfonts}

% used for TeXing text within eps files
%\usepackage{psfrag}
% need this for including graphics (\includegraphics)
%\usepackage{graphicx}
% for neatly defining theorems and propositions
%\usepackage{amsthm}
% making logically defined graphics
%%%\usepackage{xypic}

% there are many more packages, add them here as you need them

% define commands here
\begin{document}
Given a linear operator $A$, we define:
\begin{equation}
\exp{A} := \sum_{k=0}^{\infty} \frac{1}{k!}A^k.
\end{equation}
It follows that
\begin{equation}
\frac{\partial}{\partial \tau} e^{\tau A} = A e^{\tau A}= e^{\tau A} A.
\end{equation}
Consider another linear operator $B$. Let $B(\tau)=e^{\tau A} B e^{-\tau A}$. Then one can prove the following series representation for $B(\tau)$:
\begin{equation}\label{eq:genbch}
B(\tau)= \sum_{m=0}^{\infty} \frac{{\tau}^m}{m!}B_m,
\end{equation}
where $B_m =[A,B]_m := [A,[A,B]_{m-1}]$  and $B_0:=B$.
A very important special case of eq.~(\ref{eq:genbch}) is known as the 
{\sl Baker-Campbell-Hausdorff (BCH)} formula. Namely, for $\tau =1$ we get:
\begin{equation}
e^A \; B e^{-A} = \sum_{m=0}^{\infty} \frac{1}{m!} B_m.
\end{equation}
Alternatively, this expression may be rewritten as
\begin{equation}
[B,e^{-A}] = e^{-A} \left( [A,B]+\frac{1}{2} [A,[A,B]] + \cdots  
\right),
\end{equation}
or 
\begin{equation}
[e^A,B] = \left( [A,B]+\frac{1}{2} [A,[A,B]] + \cdots  
\right) e^A.
\end{equation}
There is a descendent of the BCH formula, which often is also referred to as BCH
formula. It provides us with the multiplication law for two exponentials of linear operators: Suppose $[A,[A,B]] = [B,[B,A]] = 0$. Then,
\begin{equation}
e^A e^B = e^{A+B} e^{\frac{1}{2}[A,B]}.
\end{equation}
Thus, if we want to commute two exponentials, we get an extra factor
\begin{equation}
e^A e^B = e^B e^A e^{[A,B]}.
\end{equation}
%%%%%
%%%%%
\end{document}
