\documentclass[12pt]{article}
\usepackage{pmmeta}
\pmcanonicalname{Extremal}
\pmcreated{2013-03-22 19:10:29}
\pmmodified{2013-03-22 19:10:29}
\pmowner{pahio}{2872}
\pmmodifier{pahio}{2872}
\pmtitle{extremal}
\pmrecord{4}{42082}
\pmprivacy{1}
\pmauthor{pahio}{2872}
\pmtype{Definition}
\pmcomment{trigger rebuild}
\pmclassification{msc}{47A60}
\pmsynonym{extremal curve}{Extremal}
\pmrelated{CalculusOfVariations}
\pmdefines{extremal function}

\endmetadata

% this is the default PlanetMath preamble.  as your knowledge
% of TeX increases, you will probably want to edit this, but
% it should be fine as is for beginners.

% almost certainly you want these
\usepackage{amssymb}
\usepackage{amsmath}
\usepackage{amsfonts}

% used for TeXing text within eps files
%\usepackage{psfrag}
% need this for including graphics (\includegraphics)
%\usepackage{graphicx}
% for neatly defining theorems and propositions
 \usepackage{amsthm}
% making logically defined graphics
%%%\usepackage{xypic}

% there are many more packages, add them here as you need them

% define commands here

\theoremstyle{definition}
\newtheorem*{thmplain}{Theorem}

\begin{document}
\PMlinkescapeword{extremal}

An extremum point (i.e. a maximum point or a minimum point) gives a maximum value or a minimum value for a \PMlinkname{function}{RealFunction}.\\

An \emph{extremal curve} (i.e. an \emph{extremal}, or the corresponding \emph{extremal function}) gives a maximum value or a minimum value for a \PMlinkname{functional}{Functional}.

For example, if the functional is the length of a curve connecting two points on a sphere, the extremals are great circle arcs between the points.
%%%%%
%%%%%
\end{document}
