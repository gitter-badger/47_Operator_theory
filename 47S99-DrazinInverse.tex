\documentclass[12pt]{article}
\usepackage{pmmeta}
\pmcanonicalname{DrazinInverse}
\pmcreated{2013-03-22 13:58:05}
\pmmodified{2013-03-22 13:58:05}
\pmowner{kronos}{12218}
\pmmodifier{kronos}{12218}
\pmtitle{Drazin inverse}
\pmrecord{29}{34738}
\pmprivacy{1}
\pmauthor{kronos}{12218}
\pmtype{Definition}
\pmcomment{trigger rebuild}
\pmclassification{msc}{47S99}
\pmrelated{MoorePenroseGeneralizedInverse}

% this is the default PlanetMath preamble.  as your knowledge
% of TeX increases, you will probably want to edit this, but
% it should be fine as is for beginners.

% almost certainly you want these
\usepackage{amssymb}
\usepackage{amsmath}
\usepackage{amsfonts}

% used for TeXing text within eps files
%\usepackage{psfrag}
% need this for including graphics (\includegraphics)
%\usepackage{graphicx}
% for neatly defining theorems and propositions
%\usepackage{amsthm}
% making logically defined graphics
%%%\usepackage{xypic}

% there are many more packages, add them here as you need them

% define commands here
\begin{document}
A Drazin inverse of an operator $A$ is an operator, $B$, such that
$$A B = B A,$$
$$B A B= B,$$
$$A B A= A- U,$$
where the spectral radius $r(U)=0$. The Drazin inverse ($B$) is denoted by $A^D$. It exists, if $0$ is not an accumulation point of $\sigma (A)$.

For example, a projection operator is its own Drazin inverse, $P^D=P$, as
$PPP = PP = P$; for a Shift operator $S^D=0$ holds.

The following are some other useful properties of the Drazin inverse: 
\begin{enumerate}
 \item $(A^D)^*= (A^*)^D$;
 \item $A^D= (A+ \alpha P^{(A)})^{-1} (I- P^{(A)})$, where $P^{(A)}:= I-A^D A$ is the spectral projection of $A$ at $0$ and $\alpha \neq 0$;
 \item $A^{\dagger}= (A^* A)^D A^* = A^* (A A^* )^D$, where $A^{\dagger}$ is the Moore-Penrose pseudoinverse of $A$;
 \item $A^D= A^m (A^{2m+1})^{\dagger} A^m$ for $m \ge \mbox{ind}(A)$, if      $\mbox{ind}(A):= \min \{ k: \operatorname{Im} A^k = \operatorname{Im} A^{k+1} \}$ is finite;
\item If the matrix is represented explicitly by its Jordan canonical form, ($\Lambda$ regular and $N$ nilpotent), then

$$\left( E \begin{bmatrix} \Lambda & 0 \\ 0 & N \end{bmatrix} E^{-1} \right)^D = E \begin{bmatrix} \Lambda^{-1} & 0 \\ 0 & 0 \end{bmatrix} E^{-1};$$
\item Let $e_{\lambda}^A$ denote an eigenvector of $A$ to the eigenvalue $\lambda$. Then $e_{\lambda}^A+t (\lambda I- A)^D h e_{\lambda}^A + O(t^2)$ is an eigenvector of $A+ t h$.
\end{enumerate}
%%%%%
%%%%%
\end{document}
