\documentclass[12pt]{article}
\usepackage{pmmeta}
\pmcanonicalname{BoundedInverseTheorem}
\pmcreated{2013-03-22 17:30:51}
\pmmodified{2013-03-22 17:30:51}
\pmowner{asteroid}{17536}
\pmmodifier{asteroid}{17536}
\pmtitle{bounded inverse theorem}
\pmrecord{11}{39905}
\pmprivacy{1}
\pmauthor{asteroid}{17536}
\pmtype{Corollary}
\pmcomment{trigger rebuild}
\pmclassification{msc}{47A05}
\pmclassification{msc}{46A30}
\pmsynonym{inverse mapping theorem}{BoundedInverseTheorem}

\endmetadata

% this is the default PlanetMath preamble.  as your knowledge
% of TeX increases, you will probably want to edit this, but
% it should be fine as is for beginners.

% almost certainly you want these
\usepackage{amssymb}
\usepackage{amsmath}
\usepackage{amsfonts}

% used for TeXing text within eps files
%\usepackage{psfrag}
% need this for including graphics (\includegraphics)
%\usepackage{graphicx}
% for neatly defining theorems and propositions
%\usepackage{amsthm}
% making logically defined graphics
%%%\usepackage{xypic}

% there are many more packages, add them here as you need them

% define commands here

\begin{document}
\PMlinkescapeword{near}

The next result is a corollary of the open mapping theorem. It is often called the {\bf bounded inverse theorem} or the {\bf inverse mapping theorem}.

{\bf Theorem -} Let $X, Y$ be Banach spaces. Let $T: X \longrightarrow Y$ be an invertible bounded operator. Then $T^{-1}$ is also \PMlinkescapetext{bounded}.

{\bf Proof :} $T$ is a surjective continuous operator between the Banach spaces $X$ and $Y$. Therefore, by the open mapping theorem, $T$ takes open sets to open sets.

So, for every open set $U \subseteq X$, $T(U)$ is open in $Y$.

Hence $(T^{-1})^{-1}(U)$ is open in $Y$, which proves that $T^{-1}$ is continuous, i.e. bounded. $\square$

\subsubsection{Remark}

It is usually of great importance to know if a bounded operator $T:X\longrightarrow Y$ has a bounded inverse. For example, suppose the equation
\begin{displaymath}
Tx=y
\end{displaymath}
has unique solutions $x$ for every given $y \in Y$. Suppose also that the above equation is very difficult to solve (numerically) for a given $y_0$, but easy to solve for a value $\tilde{y}$ "near" $y_0$. Then, if $T^{-1}$ is continuous,  the correspondent solutions $x_0$ and $\tilde{x}$ are also "near" since
\begin{displaymath}
\|x_0 - \tilde{x}\| = \|T^{-1}y_0 - T^{-1}\tilde{y}\| \leq \|T^{-1}\|\|y_0-\tilde{y}\|
\end{displaymath}

Therefore we can solve the equation for a "near" value $\tilde{y}$ instead, without obtaining a significant error.
%%%%%
%%%%%
\end{document}
