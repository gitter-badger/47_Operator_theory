\documentclass[12pt]{article}
\usepackage{pmmeta}
\pmcanonicalname{FugledePutnamRosenblumTheorem}
\pmcreated{2013-05-08 21:47:27}
\pmmodified{2013-05-08 21:47:27}
\pmowner{karstenb}{16623}
\pmmodifier{karstenb}{16623}
\pmtitle{Fuglede-Putnam-Rosenblum theorem}
\pmrecord{1}{}
\pmprivacy{1}
\pmauthor{karstenb}{16623}
\pmtype{Theorem}
\pmclassification{msc}{47L30}

% this is the default PlanetMath preamble.  as your knowledge
% of TeX increases, you will probably want to edit this, but
% it should be fine as is for beginners.

% almost certainly you want these
\usepackage{amssymb}
\usepackage{amsmath}
\usepackage{amsfonts}

% need this for including graphics (\includegraphics)
\usepackage{graphicx}
% for neatly defining theorems and propositions
\usepackage{amsthm}

% making logically defined graphics
%\usepackage{xypic}
% used for TeXing text within eps files
%\usepackage{psfrag}

% there are many more packages, add them here as you need them

% define commands here

\begin{document}
Let $A$ be a $C^{\ast}$-algebra with unit $e$.

The Fuglede-Putnam-Rosenblum theorem makes the assertion that for a normal element $a \in A$ the kernel of the commutator mapping $[a, -] \colon A \to A$ is a $\ast$-closed set. 

The general formulation of the result is as follows:

\textbf{Theorem.} Let $A$ be a $C^{\ast}$-algebra with unit $e$. Let two normal elements
$a, b \in A$ be given and $c \in A$ with $ac = cb$. 
Then it follows that $a^{\ast} c = c b^{\ast}$. 

\textbf{Lemma.} For any $x \in A$ we have that $\exp(x - x^{\ast})$ is a element of $A$.

\textbf{Proof.} We have for $x \in A$ that
$\exp(x - x^{\ast})^{\ast} \exp(x - x^{\ast}) = \exp(x^{\ast} - x + x - x^{\ast}) = \exp(0) = e$.
And similarly $\exp(x - x^{\ast}) \exp(x - x^{\ast})^{\ast} = e$. \qed

With this we can now give a proof the Theorem.

\textbf{Proof.} The condition $ac = cb$ implies by induction that $a^k c = c b^k$ holds for each $k \in \mathbb{N}$.
Expanding in power series on both sides yields $\exp(a) c = c \exp(b)$.
This is equivalent to $c = \exp(-a) c \exp(b)$. Set $U_1 := \exp(a^{\ast} - a), U_2 := \exp(b - b^{\ast})$. From the Lemma we obtain that $\|U_1\|_A = \|U_2\|_A = 1$. 
Since $a$ commutes with $a^{\ast}$ und $b$ with $b^{\ast}$ we obtain that
\[
\exp(a^{\ast}) c \exp(-b^{\ast}) = \exp(a^{\ast}) \exp(-a) c \exp(b) \exp(b^{\ast})
\]

which equals $\exp(a^{\ast} - a) c \exp(b - b^{\ast}) = U_1 c U_2$. 

Hence 
\[
\|\exp(a^{\ast}) c \exp(-b^{\ast})\| \leq \|c\|.
\]

Define $f \colon \mathbb{C} \to A$ by $f(\lambda) := \exp(\lambda a^{\ast}) c \exp(-\lambda b^{\ast})$. If we substitute $a \mapsto \lambda a, b \mapsto \lambda b$ in the last estimate we obtain
\[
\|f(\lambda)\| \leq \|c\|, \ \lambda \in \mathbb{C}. 
\]

But $f$ is clearly an entire function and therefore Liouville's theorem implies that $f(\lambda) = f(0) = c$ for each $\lambda$. 

This yields the equality
\[
c \exp(\lambda b^{\ast}) = \exp(\lambda a^{\ast}) c.
\]

Comparing the terms of first order for $\lambda$ small finishes the proof. \qed


\end{document}
