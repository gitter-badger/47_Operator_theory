\documentclass[12pt]{article}
\usepackage{pmmeta}
\pmcanonicalname{SelfadjointOperator}
\pmcreated{2013-03-22 13:48:23}
\pmmodified{2013-03-22 13:48:23}
\pmowner{Koro}{127}
\pmmodifier{Koro}{127}
\pmtitle{self-adjoint operator}
\pmrecord{8}{34527}
\pmprivacy{1}
\pmauthor{Koro}{127}
\pmtype{Definition}
\pmcomment{trigger rebuild}
\pmclassification{msc}{47B15}
\pmclassification{msc}{47B25}
\pmrelated{HermitianMatrix}
\pmdefines{Hermitian operator}
\pmdefines{symmetric operator}
\pmdefines{essentially self-adjoint}
\pmdefines{self-adjoint}

% this is the default PlanetMath preamble.  as your knowledge
% of TeX increases, you will probably want to edit this, but
% it should be fine as is for beginners.

% almost certainly you want these
\usepackage{amssymb}
\usepackage{amsmath}
\usepackage{amsfonts}
\usepackage{mathrsfs}

% used for TeXing text within eps files
%\usepackage{psfrag}
% need this for including graphics (\includegraphics)
%\usepackage{graphicx}
% for neatly defining theorems and propositions
%\usepackage{amsthm}
% making logically defined graphics
%%%\usepackage{xypic}

% there are many more packages, add them here as you need them

% define commands here
\newcommand{\C}{\mathbb{C}}
\newcommand{\R}{\mathbb{R}}
\newcommand{\N}{\mathbb{N}}
\newcommand{\Z}{\mathbb{Z}}
\newcommand{\Per}{\operatorname{Per}}
\begin{document}
\PMlinkescapeword{symmetric} \PMlinkescapeword{Hermitian}
A densely defined linear operator $A\colon\mathscr{D}(A)\subset \mathscr{H}\to\mathscr{H}$ on a Hilbert space $\mathscr{H}$ is a \emph{Hermitian} or \emph{symmetric} operator if $(Ax,y) = (x,Ay)$ for all $x,y\in \mathscr{D}(A)$. This means that the adjoint $A^*$ of $A$ is defined at least on $\mathscr{D}(A)$ and that its restriction to that set coincides with $A$. This fact is often denoted by $A\subset A^*$.

The operator $A$ is \emph{self-adjoint} if it coincides with its adjoint, i.e. if $A=A^*$.
If $A$ is closable and its closure coincides with its adjoint (i.e. $\overline{A}=A^*$), then $A$ is said to be \emph{essentially self-adjoint}.
%%%%%
%%%%%
\end{document}
