\documentclass[12pt]{article}
\usepackage{pmmeta}
\pmcanonicalname{ProofOfBrouwerFixedPointTheorem}
\pmcreated{2013-03-22 13:11:24}
\pmmodified{2013-03-22 13:11:24}
\pmowner{bwebste}{988}
\pmmodifier{bwebste}{988}
\pmtitle{proof of Brouwer fixed point theorem}
\pmrecord{6}{33642}
\pmprivacy{1}
\pmauthor{bwebste}{988}
\pmtype{Proof}
\pmcomment{trigger rebuild}
\pmclassification{msc}{47H10}
\pmclassification{msc}{54H25}
\pmclassification{msc}{55M20}

% this is the default PlanetMath preamble.  as your knowledge
% of TeX increases, you will probably want to edit this, but
% it should be fine as is for beginners.

% almost certainly you want these
\usepackage{amssymb}
\usepackage{amsmath}
\usepackage{amsfonts}

% used for TeXing text within eps files
%\usepackage{psfrag}
% need this for including graphics (\includegraphics)
%\usepackage{graphicx}
% for neatly defining theorems and propositions
%\usepackage{amsthm}
% making logically defined graphics
%%%\usepackage{xypic}

% there are many more packages, add them here as you need them

% define commands here
\begin{document}
Proof of the Brouwer fixed point theorem:

Assume that there does exist a map from $f:B^n\to B^n$ with no fixed point.  Then let 
$g(x)$ be the following map:  Start at $f(x)$, draw the ray going through $x$ and then let $g(x)$ be
the first intersection of that line with the sphere.  This map is continuous and well defined only
because $f$ fixes no point. Also, it is not hard to see that it must be the identity on the boundary
sphere.  Thus we have a map $g:B^n\to S^{n-1}$, which is the identity on
$S^{n-1}=\partial B^n$, that is, a retraction. Now, if $i:S^{n-1}\to B^n$ is the inclusion
map, $g\circ i=\mathrm{id}_{S^{n-1}}$.  Applying the reduced homology functor, we find that
$g_*\circ i_*=\mathrm{id}_{\tilde{H}_{n-1}(S^{n-1})}$, where $_*$ indicates the induced map on homology.

But, it is a well-known fact that $\tilde{H}_{n-1}(B^n)=0$ (since $B^n$ is contractible), and that
$\tilde{H}_{n-1}(S^{n-1})=\mathbb{Z}$.  Thus we have an isomorphism of a non-zero group onto itself
factoring through a trivial group, which is clearly impossible.  Thus we have a contradiction,
and no such map $f$ exists.
%%%%%
%%%%%
\end{document}
