\documentclass[12pt]{article}
\usepackage{pmmeta}
\pmcanonicalname{BorelFunctionalCalculus}
\pmcreated{2013-03-22 18:48:44}
\pmmodified{2013-03-22 18:48:44}
\pmowner{asteroid}{17536}
\pmmodifier{asteroid}{17536}
\pmtitle{Borel functional calculus}
\pmrecord{10}{41614}
\pmprivacy{1}
\pmauthor{asteroid}{17536}
\pmtype{Feature}
\pmcomment{trigger rebuild}
\pmclassification{msc}{47A60}
\pmclassification{msc}{46L10}
\pmclassification{msc}{46H30}
\pmrelated{FunctionalCalculus}
\pmrelated{PolynomialFunctionalCalculus}
\pmrelated{ContinuousFunctionalCalculus2}
\pmdefines{Borel functions of a normal operator}

\endmetadata

% this is the default PlanetMath preamble.  as your knowledge
% of TeX increases, you will probably want to edit this, but
% it should be fine as is for beginners.

% almost certainly you want these
\usepackage{amssymb}
\usepackage{amsmath}
\usepackage{amsfonts}

% used for TeXing text within eps files
%\usepackage{psfrag}
% need this for including graphics (\includegraphics)
%\usepackage{graphicx}
% for neatly defining theorems and propositions
%\usepackage{amsthm}
% making logically defined graphics
%%%\usepackage{xypic}

% there are many more packages, add them here as you need them

% define commands here

\begin{document}
\PMlinkescapeword{bounded}
\PMlinkescapeword{spectrum}
\PMlinkescapeword{properties}
\PMlinkescapeword{property}
\PMlinkescapeword{closure}
\PMlinkescapeword{strong}

Let $B(H)$ be the \PMlinkname{algebra}{Algebra} of bounded operators over a complex Hilbert space $H$ and $T \in B(H)$ a normal operator.

The {\bf Borel functional calculus} is a functional calculus which enables the expression
\begin{align*}
f(T)
\end{align*}
to make sense as a bounded operator in $H$, for a \PMlinkname{bounded}{Bounded} Borel function $f$.

In particular, it allows the definition of operators $\chi_S (T)$ for any characteristic function $\chi_S$, which are of significant importance on the \PMlinkescapetext{comprehension} of the \PMlinkescapetext{structure} of $T$.

The Borel functional calculus will be constructed by extending the continuous functional calculus for arbitrary bounded Borel functions.

\section{Preliminary Facts}

Let us set some notation first:
\begin{itemize}
\item $\sigma(T)$ will denote the \PMlinkname{spectrum}{Spectrum} of $T$.
\item $C(\sigma(T))$ will denote the \PMlinkname{$C^*$-algebra}{CAlgebra} of continuous functions $\sigma(T) \to \mathbb{C}$.
\item $B(\sigma(T))$ will denote the $C^*$-algebra of bounded Borel functions $\sigma(T) \to \mathbb{C}$, endowed with the sup norm.
\end{itemize}

The continuous functional calculus for $T$ allows the expression $f(T)$ to make sense for continuous functions $f \in C(\sigma(T))$, by the assignment of a unital *-homomorphism
\begin{align*}
\pi: C(\sigma(T)) \longrightarrow B(H) \\
f \longmapsto f(T):= \pi(f)
\end{align*}
that sends the identity function to $T$. This unital *-homomorphism is in fact uniquely determined by this property (see the entry on the \PMlinkname{continuous functional calculus}{ContinuousFunctionalCalculus2} for more details).

The objective is to extend $\pi$ to a unital *-homomorphism $\widetilde{\pi}: B(\sigma(T)) \longrightarrow B(H)$.

Since $B(\sigma(T))$ is a much larger $C^*$-algebra than $C(\sigma(T))$, there is no reson to presume that there is only one extension of $\pi$. Which extension would be the most natural then? It turns out that there is a unique extension that satisfies a good continuity property.

It is known that *-homomorphisms between $C^*$-algebras are continuous (see \PMlinkname{this entry}{HomomorphismsOfCAlgebrasAreContinuous}), so that whenever a net $f_i \in B(\sigma(T))$ converges in the sup norm to a function $f \in B(\sigma(T))$ we will have that $f_i(T) \to f(T)$ in the operator norm. All extensions of $\pi$ will automaticaly satisfy this continuity property, but this can be improved in a satisfactory manner.

$\,$

{\bf Notation -} \emph{Let $X$ be a compact Hausdorff space, $M(X)$ the space of all finite \PMlinkname{regular}{OuterRegular} Borel measures in $X$ and $B(X)$ the $C^*$-algebra of all bounded Borel functions in $X$. The weakest topology in $B(X)$ for which integration against any measure $\nu$ is continuous will be reffered to as the} $\mu$-topology. \emph{This means that $f_i \to f$ in the $\mu$-topology if and only if $\int f_i d\nu \to \int f d\nu$ for all $\nu \in M(X)$.}

$\,$

Notice that we can identify each function $f \in B(X)$ with a bounded \PMlinkname{linear functional}{Functional} $\omega_f$ in $M(X)$, given by
\begin{align*}
\omega_f (\nu):= \int_X f d \nu\,, \qquad\qquad \nu \in M(X)
\end{align*}
and the $\mu$-topology corresponds exactly to the weak-* topology under this identification.

We will see in the next \PMlinkescapetext{section} that there is an unique extension of $\pi$ that is continuous from the $\mu$-topology to the weak operator topology.

Just like the \PMlinkname{Stone-Weierstrass theorem}{StoneWeierstrassTheoremComplexVersion} allowed the passage from the polynomial functional calculus to the continuous functional calculus, the \PMlinkname{Riesz representation theorem}{RieszRepresentationTheoremOfLinearFunctionalsOnFunctionSpaces} will allow the passage from the latter to the Borel functional calculus.

\section{Definition}

The following result is the key for the definition of the Borel functional calculus.

$\,$

{\bf Theorem 1 -} \emph{Let $T$ be a normal operator in $B(H)$ and $\pi:C(\sigma(T)) \longrightarrow B(H)$ the unital *-homomorphism corresponding to the continuous functional calculus for $T$. Then, $\pi$ extends uniquely to a *-homomorphism $\widetilde{\pi}: B(\sigma(T)) \longrightarrow B(H)$ that is continuous from the $\mu$-topology to the weak operator topology. Moreover, each operator $\pi(f)$ lies in \PMlinkname{strong operator}{OperatorTopologies} \PMlinkname{closure}{Closure} of the unital *-algebra generated by $T$.}

$\,$

{\bf \emph{\PMlinkescapetext{Proof}:}} See \PMlinkname{this attached entry}{ProofOfBorelFunctionalCalculus}

$\,$

We are now able to define the Borel functional calculus:

{\bf Definition -} \emph{Let $T$ be a normal operator in $B(H)$. Let $\widetilde{\pi}: B(\sigma(T)) \longrightarrow B(H)$ be the unique *-homomorphism defined in Theorem 1. This *-homomorphism is denoted by}
\begin{align*}
f \longmapsto f(T)\,, \qquad\qquad f \in B(\sigma(T))
\end{align*}
\emph{and it is called the} {\bf Borel functional calculus} \emph{for $T$}.

$\,$

Since this functional calculus extends the polynomial functional calculus, we have that for any polynomial $p(z):= \sum c_{n,m} z^n \overline{z}^m$,
\begin{align*}
p(T)= \sum c_{n,m} T^n(T^*)^m
\end{align*}
Moreover, since $f(T)$ lies in the strong operator closure of the unital *-algebra generated by $T$, for any function $f \in B(\sigma(T))$, we see that $f(T)$ is the strong operator limit of polynomials $\sum c_{n,m} T^n(T^*)^m$.

\section{Borel Calculus in von Neumann Algebras}

The Borel functional calculus is in fact applicable for any normal operator $T$ in any von Neumann algebra $\mathcal{M}$.

That is due to the fact, expressed in Theorem 1, that for every $f \in B(\sigma(T))$ the operator $f(T)$ belongs to the strong operator closure of the unital *-algebra generated by $T$. Being a von Neumann algebra, $\mathcal{M}$ is \PMlinkname{closed}{ClosedSet} in the strong operator topology, and therefore all operators $f(T)$ belong to $\mathcal{M}$.

Thus, by restriction, we have in fact a *-homomorphism
\begin{align*}
\widetilde{\pi}: B(\sigma(T)) \longrightarrow \mathcal{M} \\
f \longmapsto f(T)
\end{align*}
satisfying the properties of Theorem 1, i.e. we have a Borel functional calculus for normal operators of a von Neumann algebra.

\begin{thebibliography}{9}
\bibitem{Arv} W. Arveson, \emph{A Short Course on Spectral Theory}, Graduate Texts in Mathematics, 209, Springer, New York, 2002
\bibitem{Weaver} N. Weaver, \emph{Mathematical Quantization}, Studies in Advanced Mathematics, Chapman \& Hall/CRC, Boca Raton, FL, 2001
\end{thebibliography}
%%%%%
%%%%%
\end{document}
