\documentclass[12pt]{article}
\usepackage{pmmeta}
\pmcanonicalname{DunklWilliamsInequality}
\pmcreated{2013-03-22 16:56:38}
\pmmodified{2013-03-22 16:56:38}
\pmowner{Mathprof}{13753}
\pmmodifier{Mathprof}{13753}
\pmtitle{Dunkl-Williams inequality}
\pmrecord{12}{39212}
\pmprivacy{1}
\pmauthor{Mathprof}{13753}
\pmtype{Definition}
\pmcomment{trigger rebuild}
\pmclassification{msc}{47A12}

\endmetadata

% this is the default PlanetMath preamble.  as your knowledge
% of TeX increases, you will probably want to edit this, but
% it should be fine as is for beginners.

% almost certainly you want these
\usepackage{amssymb}
\usepackage{amsmath}
\usepackage{amsfonts}

% used for TeXing text within eps files
%\usepackage{psfrag}
% need this for including graphics (\includegraphics)
%\usepackage{graphicx}
% for neatly defining theorems and propositions
%\usepackage{amsthm}
% making logically defined graphics
%%%\usepackage{xypic}

% there are many more packages, add them here as you need them

% define commands here

\begin{document}
Let $V$ be an inner product space and $a,b \in V$. If $a \not = 0$ and $b \not = 0$, then 
\begin{equation}
\Vert  a -b \Vert \ge \frac{1}{2}(  \Vert a   \Vert + \Vert b \Vert ) \Vert \frac{a}{\Vert a \Vert} - \frac{b}{\Vert b \Vert} \Vert .
\end{equation}
Equality holds if and only if $a=0$, $b=0$, $\Vert a \Vert = \Vert b \Vert$ 
or $a \Vert b \Vert = b \Vert a \Vert $.
In fact, if (1) holds
and $V$ is a normed linear space, then $V$ is an inner product space.

If $X$ is a normed linear space and $a \not =0$ and $b \not = 0$ then
\begin{equation}
\Vert  a -b \Vert \ge \frac{1}{4}(  \Vert a   \Vert + \Vert b \Vert ) \Vert \frac{a}{\Vert a \Vert} - \frac{b}{\Vert b \Vert} \Vert .
\end{equation}
Equality holds if and only if $a=0$,  $b=0$ or $a=b$.
The constant $\frac{1}{4}$ is best possible. For example, let  $X$ be  the set of  ordered pairs of real numbers,
with norm of $(x_1, x_2)$ equal to $\vert x_1 \vert  + \vert x_2 \vert.$ Let
$a=(1, \epsilon)$ and $b=(1,0)$ where $\epsilon$ is a small positive number. After a bit 
of routine calculation, it is easily seen that the best possible constant is $\frac{1}{4}$.

The inequality (2) has been generalized in the case where $X$ is a normed linear space
over the reals.  In that case one can show:
\begin{equation}
\Vert  a -b \Vert \ge c_p  (  \Vert a   \Vert^p + \Vert b \Vert^p )^{1/p} \Vert \frac{a}{\Vert a \Vert} - \frac{b}{\Vert b \Vert} \Vert 
\end{equation}
where $c_p = 2^{-1-1/p}$ if $0 < p \le 1$ and $c_p = 1/4$ if $p \ge 1$. The case $p=1$ is the
Dunkl and Williams inequality. 

If $X$ is a normed linear space  and $0 < p \le 1$ then (3) holds with $c_p = 2^{-1/p}$ 
if and only if $X$ is an inner product space.

The inequality (2) can be improved slightly to get:
\begin{equation}
\Vert  a -b \Vert \ge \frac{1}{2} \max(\Vert a \Vert, \Vert b \Vert)  \Vert \frac{a}{\Vert a \Vert} - \frac{b}{\Vert b \Vert} \Vert .
\end{equation}
Equality holds in (4) if and only if $a$ and $b$ span an ${\ell_2}^1$ in the underlying real vector space with
$\pm\Vert b-a\Vert^{-1}(b-a)$ and $\pm\Vert a\Vert^{-1}a$ (or $\pm\Vert b\Vert^{-1}b$) as the vertices of the unit
parallelogram.







\begin{thebibliography}{99}
\bibitem{DW} C.F. Dunkl, K.S. Williams, \emph{A simple norm inequality}. Amer. Math. Monthly, \textbf{71} (1) (1964) 53-54.
\bibitem{KS} W.A. Kirk, M.F.  Smiley, \emph{Another characterization of inner product spaces}, Amer. Math. Monthly,
\textbf{71}, (1964), 890-891.
\bibitem{AMA} A.M. Alrashed, \emph{Norm inequalities and Characterizations of Inner Product Spaces},
J. Math. Anal. Appl. \textbf{176}, (2) (1993), 587-593.
\bibitem{MS} J.L. Massera, J.J. Sch\"affer, \emph{Linear differential equations and functional analysis}, Annals of Math. \textbf{67} (2)(1958), 517-573. (on page 538)
\bibitem{LMK)}
L.M. Kelly, \emph{The Massera-Sch\"aeffer equality}, Amer. Math. Monthly, \textbf{73}, (1966) 1102-1103.
\end{thebibliography}

%%%%%
%%%%%
\end{document}
