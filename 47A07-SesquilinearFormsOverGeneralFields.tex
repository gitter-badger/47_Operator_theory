\documentclass[12pt]{article}
\usepackage{pmmeta}
\pmcanonicalname{SesquilinearFormsOverGeneralFields}
\pmcreated{2013-03-22 15:58:17}
\pmmodified{2013-03-22 15:58:17}
\pmowner{Algeboy}{12884}
\pmmodifier{Algeboy}{12884}
\pmtitle{sesquilinear forms over general fields}
\pmrecord{11}{37987}
\pmprivacy{1}
\pmauthor{Algeboy}{12884}
\pmtype{Definition}
\pmcomment{trigger rebuild}
\pmclassification{msc}{47A07}
\pmclassification{msc}{15A63}
\pmclassification{msc}{11E39}
\pmclassification{msc}{51A05}
\pmsynonym{Hermitian form}{SesquilinearFormsOverGeneralFields}
\pmsynonym{Hermitean form}{SesquilinearFormsOverGeneralFields}
%\pmkeywords{sesquilinear form}
%\pmkeywords{Hermitian form}
\pmrelated{ReflexiveNonDegenerateSesquilinear}
\pmrelated{NonDegenerate}
\pmrelated{Polarity2}
\pmrelated{Projectivity}
\pmrelated{ProjectiveGeometry}
\pmrelated{Isometry2}
\pmrelated{ProjectiveGeometry3}
\pmrelated{ClassicalGroups}
\pmdefines{sesquilinear form}
\pmdefines{Hermitian form}
\pmdefines{bilinear form}
\pmdefines{Hermitean}

\usepackage{latexsym}
\usepackage{amssymb}
\usepackage{amsmath}
\usepackage{amsfonts}
\usepackage{amsthm}

%%\usepackage{xypic}

%-----------------------------------------------------

%       Standard theoremlike environments.

%       Stolen directly from AMSLaTeX sample

%-----------------------------------------------------

%% \theoremstyle{plain} %% This is the default

\newtheorem{thm}{Theorem}

\newtheorem{coro}[thm]{Corollary}

\newtheorem{lem}[thm]{Lemma}

\newtheorem{lemma}[thm]{Lemma}

\newtheorem{prop}[thm]{Proposition}

\newtheorem{conjecture}[thm]{Conjecture}

\newtheorem{conj}[thm]{Conjecture}

\newtheorem{defn}[thm]{Definition}

\newtheorem{remark}[thm]{Remark}

\newtheorem{ex}[thm]{Example}



%\countstyle[equation]{thm}



%--------------------------------------------------

%       Item references.

%--------------------------------------------------


\newcommand{\exref}[1]{Example-\ref{#1}}

\newcommand{\thmref}[1]{Theorem-\ref{#1}}

\newcommand{\defref}[1]{Definition-\ref{#1}}

\newcommand{\eqnref}[1]{(\ref{#1})}

\newcommand{\secref}[1]{Section-\ref{#1}}

\newcommand{\lemref}[1]{Lemma-\ref{#1}}

\newcommand{\propref}[1]{Prop\-o\-si\-tion-\ref{#1}}

\newcommand{\corref}[1]{Cor\-ol\-lary-\ref{#1}}

\newcommand{\figref}[1]{Fig\-ure-\ref{#1}}

\newcommand{\conjref}[1]{Conjecture-\ref{#1}}


% Normal subgroup or equal.

\providecommand{\normaleq}{\unlhd}

% Normal subgroup.

\providecommand{\normal}{\lhd}

\providecommand{\rnormal}{\rhd}
% Divides, does not divide.

\providecommand{\divides}{\mid}

\providecommand{\ndivides}{\nmid}


\providecommand{\union}{\cup}

\providecommand{\bigunion}{\bigcup}

\providecommand{\intersect}{\cap}

\providecommand{\bigintersect}{\bigcap}










\begin{document}
Let $V$ be a vector space over a field $k$.  $k$ may be of any characteristic.

\section{Sesquilinear Forms}

\begin{defn}
A function $b:V\times V\rightarrow k$ is sesquilinear if it satisfies each of
the following:
\begin{enumerate}
\item $b(v,w+u)=b(v,w)+b(v,u)$ and $b(v+u,w)=b(v,w)+b(u,w)$ for all $u,v,w\in V$;
\item For a given field automorphism $\theta$ of $k$, $b(v,lw)=l^\theta b(v,w)$ and $b(lv,w)=lb(v,w)$ for all $v,w\in V$ and $l\in k$.
\end{enumerate}
\end{defn}

\begin{remark}
It is possible to apply the field automorphism in the first variable but is more common to do so in the second variable.  Also, if $\theta=1$ the form is a bilinear form.  
\end{remark}

Sesquilinear forms are commonly ascribed any combination of the following properties:
\begin{itemize}
\item non-degenerate,
\item reflexive, (commonly required to define perpendicular);
\item positive definite (this condition requires the fixed field of $\theta$,
$k_0$, be an ordered field, such as the rationals $\mathbb{Q}$ or reals $\mathbb{R}$).
\end{itemize}

Non-degenerate sesquilinear and bilinear forms apply to projective geometries as dualities and polarities through the induced $\perp$ operation.  (See \PMlinkname{polarity}{Polarity2}.)

\section{Hermitian Forms}

If $\theta^2=1$, it is common to exchange notation at this point and use the same notation of $\bar{l}$ for $l^\theta$ as is common for complex conjugation -- even if $k$ is not $\mathbb{C}$.  Then $\bar{\bar{l}}=l$.

In this notation, Hermitian forms may be defined by the property
\[b(v,w)=\overline{b(w,v)}.\]

\begin{remark}
It is not uncommon to see hermitian or Hermitean instead of Hermitian.  The name is a tribute to Charles Hermite of the Ecole Polytechnique.
\end{remark}


%%%%%
%%%%%
\end{document}
