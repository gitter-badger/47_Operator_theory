\documentclass[12pt]{article}
\usepackage{pmmeta}
\pmcanonicalname{SpectralMeasure}
\pmcreated{2013-03-22 17:32:06}
\pmmodified{2013-03-22 17:32:06}
\pmowner{asteroid}{17536}
\pmmodifier{asteroid}{17536}
\pmtitle{spectral measure}
\pmrecord{11}{39932}
\pmprivacy{1}
\pmauthor{asteroid}{17536}
\pmtype{Definition}
\pmcomment{trigger rebuild}
\pmclassification{msc}{47A56}
\pmclassification{msc}{46G12}
\pmclassification{msc}{46G10}
\pmclassification{msc}{28C20}
\pmclassification{msc}{28B05}
\pmsynonym{projection valued measure}{SpectralMeasure}
\pmdefines{integration against spectral measures}

% this is the default PlanetMath preamble.  as your knowledge
% of TeX increases, you will probably want to edit this, but
% it should be fine as is for beginners.

% almost certainly you want these
\usepackage{amssymb}
\usepackage{amsmath}
\usepackage{amsfonts}

% used for TeXing text within eps files
%\usepackage{psfrag}
% need this for including graphics (\includegraphics)
%\usepackage{graphicx}
% for neatly defining theorems and propositions
%\usepackage{amsthm}
% making logically defined graphics
%%%\usepackage{xypic}

% there are many more packages, add them here as you need them

% define commands here

\begin{document}
\PMlinkescapeword{projection}
\PMlinkescapeword{projections}
\PMlinkescapeword{orthogonal}
\PMlinkescapeword{property}

\section{Definition}

In this entry by a \emph{projection} we \PMlinkescapetext{mean} an orthogonal projection over some Hilbert space. Also, we say that two projections are \emph{orthogonal} if their images are orthogonal subspaces.

Let $H$ be an Hilbert space, $B(H)$ the algebra of bounded operators in $H$ and $(X, \mathcal{B})$ a measurable space.

{\bf Definition -} \emph{A {\bf spectral measure} in $X$ is a function $P : \mathcal{B} \longrightarrow B(H)$ such that}
\begin{itemize}
\item[] {\bf a)} \emph{$P(E)$ is a projection in $B(H)$ for every $E \in \mathcal{B}$.}
\item[] {\bf b)} \emph{$P(\emptyset)=0$.}
\item[] {\bf c)} \emph{$P(X) = I$, where $I$ denotes the identity operator in $B(H)$.}
\item[] {\bf d)} \emph{If $E_1$ and $E_2$ are disjoint subsets of $\mathcal{B}$, then $P(E_1)$ and $P(E_2)$ are orthogonal.}
\item[] {\bf e)} \emph{$\displaystyle P(\bigcup_{n=1}^{\infty} E_n) = \sum_{n=1}^{\infty} P(E_n)\;\;$ for every sequence $E_1, E_2, \ldots$ of disjoint sets in $\mathcal{B}$.}
\end{itemize}

The \PMlinkescapetext{series} in the last condition is interpreted as the pointwise limit of the partial sums. Since from condition (d) the projections $P(E_1), P(E_2), \ldots$ are orthogonal, we know that the pointwise limit exists and is a projection (see \PMlinkname{this entry}{LatticeOfProjections}, Theorem 5).

$\,$

{\bf \PMlinkescapetext{Properties} :} In the following $\mathrm{Ran}(T)$ denotes the \PMlinkname{range}{Function} of an operator $T \in B(H)$. 
\begin{itemize}
\item $E_1 \subseteq E_2\; \Longrightarrow \; \mathrm{Ran}(P(E_1)) \subseteq \mathrm{Ran}(P(E_2))$.
\item $P(E_1 \cap E_2) = P(E_1)P(E_2)\;\;$ for every $E_1, E_2 \in \mathcal{B}$.
\end{itemize}

Thus, a spectral measure is a countably additive vector measure whose values are projections. For that, spectral measures are also called \emph{projection valued measures}.

\section{Examples}

\begin{itemize}
\item Let $(X, \mathcal{B}, \mu)$ be a measure space. Consider the Hilbert space \PMlinkname{$L^2(X, \mu)$}{L2SpacesAreHilbertSpaces}. We regard a function $f$ in \PMlinkname{$L^{\infty}(X, \mu)$}{LpSpace} as the multiplication operator $M_f \in B(L^2(X,\mu))$ given by
\begin{align*}
M_f (\xi) = f\xi\,, \qquad\qquad \xi \in L^2(X, \mu)
\end{align*}
In this setting, the characteristic functions are projections in $B(L^2(X, \mu))$ and we have a spectral measure given by
\begin{align*}
P:X & \longrightarrow  B(L^2(X, \mu)) \\
P(E) & :=  \chi_E
\end{align*}
\item Let $H$ be a Hilbert space, $T \in B(H)$ a normal operator and $\sigma(T)$ the spectrum of $T$. For any measurable subset $E \subseteq \sigma(T)$ the operators $\chi_E (T)$, given by the Borel functional calculus, are projections in $B(H)$. Moreover, we have a spectral measure given by:
\begin{align*}
P:X & \longrightarrow  B(H) \\
P(E) & :=  \chi_E(T)
\end{align*}
\end{itemize}

\section{Equivalent Definition}

The following result provides a very useful equivalent definition of a spectral measure.

{\bf Theorem 1 -} \emph{A function $P: \mathcal{B} \longrightarrow B(H)$ whose values are projections is a spectral measure in $X$ if and only if $P(X) = I$ and for every $\xi, \eta \in H$ the function $\mu_{\xi, \eta}: X \longrightarrow \mathbb{C}$ given by}
\begin{align*}
\mu_{\xi, \eta}(E) := \langle P(E)\xi, \eta \rangle
\end{align*}
\emph{is a complex measure in $X$.}

\section{Integration against spectral measures}

Let $f :X \longrightarrow \mathbb{C}$ be a \PMlinkname{bounded}{Bounded} measurable function and $P$ a spectral measure in $X$. We are interested to give meaning to the integral
\begin{displaymath}
\int_X f dP
\end{displaymath}

Since we are dealing with ``measures'' whose values are linear operators it is reasonable to expect that this integral is itself a linear operator.

There are two natural ways to define it that turn out to be equivalent. The first approach is a construction that resembles the approximation of $f$ by simple functions in Lebesgue integral theory. Here the role of simple functions will be played by the operators of the form
\begin{align*}
\sum_i \lambda_i\, P(E_i)\,, \qquad\qquad \lambda_i \in \mathbb{C}
\end{align*}

{\bf Theorem 2 -} \emph{There exists a unique operator $S \in B(H)$ with the following property: for any given $\epsilon >0$ and for every measurable partition $\{E_1, \cdots, E_n\}$ of $X$ that satisfies $|f(x)-f(x')| < \epsilon$ for all $x, x' \in E_i$, we have}
\begin{align*}
\|S - \sum_{i=1}^n f(x_i) P(E_i)\| < \epsilon
\end{align*}
\emph{for any choice of points $x_i \in E_i$.}

$\,$

We can then define $\displaystyle \int_X f dP$ as the unique operator $S$ described by Theorem 2.

The other approach to define this integral is by specifying an appropriate bounded sesquilinear form. Recall that from \PMlinkname{Riesz representation theorem}{RieszRepresentationTheoremOfBoundedSesquilinearForms}, to every bounded sesquilinear form corresponds a unique bounded operator. The construction is as follows:

First we notice that, from the alternative defintion of spectral measure (Theorem 1), for every vectors $\xi, \eta \in H$ we can define a complex measure $\mu_{\xi, \eta}$ by
\begin{displaymath}
\mu_{\xi, \eta}(E) = \langle P(E)\,\xi, \eta \rangle,
\end{displaymath}
whose total variation is estimated by $\|\mu_{\xi, \eta}\| \leq \|\xi\| \|\eta\|$.

Then we notice that the function $[\cdot,\cdot] : H \times H \longrightarrow \mathbb{C}$ defined by
\begin{displaymath}
[\xi, \eta] :=\int_X f\; d\mu_{\xi, \eta}
\end{displaymath}
is a \PMlinkescapetext{bounded} sesquilinear form.

Then, by the \PMlinkname{Riesz representation theorem}{RieszRepresentationTheoremOfBoundedSesquilinearForms}, there exists a unique operator $S \in B(H)$ such that
\begin{align}
\langle S \xi, \eta \rangle = \int_X f\; d\mu_{\xi, \eta}, \quad\quad \xi, \eta \in H
\end{align}

We can then define $\displaystyle \int_X f \;dP$ as this operator $S$. Of course, the two definitions are equivalent. We summarize this in the following result

$\,$

{\bf Theorem 3 -} \emph{Given a spectral measure $P$ and a bounded Borel function $f$, an operator $S$ that satisfies condition (1) also satisfies the conditions of Theorem 2. Therefore, both definitions of the integral of $f$ with respect to $P$ coincide and we have that:}
\begin{itemize}
\item $\displaystyle \big\langle \int_X f\; dP\, \xi, \eta \big\rangle = \int_X f \;d \mu_{\xi, \eta}$
\item $\displaystyle \int_X f\; dP$ \emph{can be arbitrarilly approximated in norm by operators of the form} $\displaystyle \sum_{i=1}^n f(x_i)P(E_i)$.
\end{itemize}


\section{Remarks}

The second example we gave above, of a spectral measure associated with a normal operator, is in some sense the general case: all spectral projections in $\mathbb{C}$ supported in a compact set arise from a normal operator. Thus, to any such spectral projection we can associate a normal operator and vice-versa. This interplay between spectral projections and normal operators is deeply explored in some versions of the spectral theorem.

\begin{thebibliography}{9}
\bibitem{Arv} W. Arveson, \emph{A Short Course on Spectral Theory}, Graduate Texts in Mathematics, 209, Springer, New York, 2002
\bibitem{conway} J. B. Conway, \emph{A Course in Functional Analysis}, 2nd ed., Graduate Texts in Mathematics, 96, Springer-Verlag, New York, Berlin, 1990.
\end{thebibliography}
%%%%%
%%%%%
\end{document}
