\documentclass[12pt]{article}
\usepackage{pmmeta}
\pmcanonicalname{PropertiesOfTheAdjointOperator}
\pmcreated{2013-03-22 13:48:14}
\pmmodified{2013-03-22 13:48:14}
\pmowner{Koro}{127}
\pmmodifier{Koro}{127}
\pmtitle{properties of the adjoint operator}
\pmrecord{12}{34524}
\pmprivacy{1}
\pmauthor{Koro}{127}
\pmtype{Theorem}
\pmcomment{trigger rebuild}
\pmclassification{msc}{47A05}

\endmetadata

% this is the default PlanetMath preamble.  as your knowledge
% of TeX increases, you will probably want to edit this, but
% it should be fine as is for beginners.

% almost certainly you want these
\usepackage{amssymb}
\usepackage{amsmath}
\usepackage{amsfonts}
\usepackage{mathrsfs}

% used for TeXing text within eps files
%\usepackage{psfrag}
% need this for including graphics (\includegraphics)
%\usepackage{graphicx}
% for neatly defining theorems and propositions
%\usepackage{amsthm}
% making logically defined graphics
%%%\usepackage{xypic}

% there are many more packages, add them here as you need them

% define commands here
\newtheorem{proposition}{Proposition}
\newcommand{\C}{\mathbb{C}}
\newcommand{\R}{\mathbb{R}}
\newcommand{\N}{\mathbb{N}}
\newcommand{\Z}{\mathbb{Z}}
\newcommand{\Per}{\operatorname{Per}}
\begin{document}
Let $A$ and $B$ be linear operators in a Hilbert space, 
and let $\lambda\in \C$. Assuming all the operators involved are densely defined, the following properties hold:
\begin{enumerate}
\item If $A^{-1}$ exists and is densely defined, then $(A^{-1})^* = (A^*)^{-1}$;
\item $(\lambda A)^* = \overline{\lambda}A^*$;
\item $A\subset B$ implies $B^*\subset A^*$;
\item $A^*+B^*\subset (A+B)^*$;
\item $B^*A^*\subset (AB)^*$;
\item $(A+ \lambda I)^* = A^*+\overline{\lambda}I$;
\item $A^*$ is a closed operator.
\end{enumerate}

\textbf{Remark.} The notation $A\subset B$ for operators means that $B$ is an  \PMlinkescapetext{extension} of $A$, i.e. $A$ is the 
\PMlinkname{restriction}{RestrictionOfAFunction} of $B$ to a smaller domain.

Also, we have the following

\begin{proposition}
If $A$ admits a \PMlinkname{closure}{ClosedOperator} $\overline{A}$, then $A^*$ is densely defined and $(A^*)^* = \overline{A}$.
\end{proposition}
%%%%%
%%%%%
\end{document}
