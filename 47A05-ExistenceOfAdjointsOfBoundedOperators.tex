\documentclass[12pt]{article}
\usepackage{pmmeta}
\pmcanonicalname{ExistenceOfAdjointsOfBoundedOperators}
\pmcreated{2013-03-22 17:33:44}
\pmmodified{2013-03-22 17:33:44}
\pmowner{asteroid}{17536}
\pmmodifier{asteroid}{17536}
\pmtitle{existence of adjoints of bounded operators}
\pmrecord{4}{39971}
\pmprivacy{1}
\pmauthor{asteroid}{17536}
\pmtype{Theorem}
\pmcomment{trigger rebuild}
\pmclassification{msc}{47A05}
\pmsynonym{bounded operators have (bounded) adjoints}{ExistenceOfAdjointsOfBoundedOperators}

\endmetadata

% this is the default PlanetMath preamble.  as your knowledge
% of TeX increases, you will probably want to edit this, but
% it should be fine as is for beginners.

% almost certainly you want these
\usepackage{amssymb}
\usepackage{amsmath}
\usepackage{amsfonts}

% used for TeXing text within eps files
%\usepackage{psfrag}
% need this for including graphics (\includegraphics)
%\usepackage{graphicx}
% for neatly defining theorems and propositions
%\usepackage{amsthm}
% making logically defined graphics
%%%\usepackage{xypic}

% there are many more packages, add them here as you need them
\usepackage{mathrsfs}

% define commands here

\begin{document}
\PMlinkescapeword{bounded}

Let $\mathscr{H}$ be a Hilbert space and let $T: \mathscr{D}(T)\subset \mathscr{H}\longrightarrow \mathscr{H}$ be a densely defined linear operator.

{\bf Theorem - } If $T$ is \PMlinkname{bounded}{ContinuousLinearMapping} then its adjoint $T^*$ is everywhere defined and is also bounded.

{\bf Proof :} Since $T$ is densely defined and bounded, it extends uniquely to a bounded (everywhere defined) linear operator on $\mathscr{H}$, which we denote by $\widetilde{T}$.

For each $z \in \mathscr{H}$, the function $f: \mathscr{H} \longrightarrow \mathbb{C}$ defined by 
$f(x)=\langle \widetilde{T}x, z\rangle$ defines a bounded linear functional on $\mathscr{H}$. By the Riesz representation theorem there exists $u \in \mathscr{H}$ such that

\begin{displaymath}
f(x) = \langle x, u \rangle
\end{displaymath}

i.e.

\begin{displaymath}
\langle \widetilde{T}x, z\rangle = \langle x, u \rangle .
\end{displaymath}

Since $\widetilde{T}$ extends $T$, we also have that for every $z \in \mathscr{H}$ there exists $u \in \mathscr{H}$ such that

\begin{displaymath}
\langle Tx, z\rangle = \langle x, u\rangle \;\;\text{for every}\; x \in \mathscr{D}(T) .
\end{displaymath}

We conclude that $T^*$ is everywhere defined. To see that it is bounded one just needs to check that

\begin{displaymath}
\sup_{z\neq 0} \frac{\|T^*z\|}{\|z\|} = \sup_{\substack{z\neq 0 \\ T^*z \neq 0}}\; \frac{|\langle T^*z, T^*z \rangle |}{\|T^*z\| \|z\|} \leq \sup_{\substack{z \neq  0 \\ x \neq 0}}\; \frac{|\langle x, T^*z \rangle |}{\|x\| \|z\|} =
\sup_{\substack{z \neq  0 \\ x \neq 0}}\; \frac{|\langle Tx, z \rangle |}{\|x\| \|z\|} \leq \|T\|
\end{displaymath}

where the last inequality comes from the Cauchy-Schwarz inequality and the fact that $T$ is bounded. $\square$

{\bf Remark -} This theorem shows in particular that bounded linear operators $T : \mathscr{H} \longrightarrow \mathscr{H}$ have bounded adjoints $T^* : \mathscr{H} \longrightarrow \mathscr{H}$.
%%%%%
%%%%%
\end{document}
