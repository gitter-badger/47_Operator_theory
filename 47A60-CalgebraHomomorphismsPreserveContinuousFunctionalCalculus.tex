\documentclass[12pt]{article}
\usepackage{pmmeta}
\pmcanonicalname{CalgebraHomomorphismsPreserveContinuousFunctionalCalculus}
\pmcreated{2013-03-22 18:00:50}
\pmmodified{2013-03-22 18:00:50}
\pmowner{asteroid}{17536}
\pmmodifier{asteroid}{17536}
\pmtitle{$C^*$-algebra homomorphisms preserve continuous functional calculus}
\pmrecord{5}{40529}
\pmprivacy{1}
\pmauthor{asteroid}{17536}
\pmtype{Theorem}
\pmcomment{trigger rebuild}
\pmclassification{msc}{47A60}
\pmclassification{msc}{46L05}

\endmetadata

% this is the default PlanetMath preamble.  as your knowledge
% of TeX increases, you will probably want to edit this, but
% it should be fine as is for beginners.

% almost certainly you want these
\usepackage{amssymb}
\usepackage{amsmath}
\usepackage{amsfonts}

% used for TeXing text within eps files
%\usepackage{psfrag}
% need this for including graphics (\includegraphics)
%\usepackage{graphicx}
% for neatly defining theorems and propositions
%\usepackage{amsthm}
% making logically defined graphics
%%%\usepackage{xypic}

% there are many more packages, add them here as you need them

% define commands here

\begin{document}
Let us setup some notation first: Let $\mathcal{A}$ be a unital \PMlinkname{$C^*$-algebra}{CAlgebra} and $z$ a normal element of  $\mathcal{A}$. Then
\begin{itemize}
\item $\sigma(z)$ denotes the spectrum of $z$.
\item $C(\sigma(z))$ denotes the $C^*$-algebra of continuous functions $\sigma(z) \longrightarrow \mathbb{C}$.
\item If $f \in C(\sigma(z))$ then $f(z)$ is the element of $\mathcal{A}$ given by the continuous functional calculus.
\end{itemize}

$\,$

{\bf Theorem -} Let $\mathcal{A}$, $\mathcal{B}$ be unital \PMlinkname{$C^*$-algebras}{CAlgebra} and $\Phi :\mathcal{A} \longrightarrow \mathcal{B}$ a *-homomorphism. Let $x$ be a normal element in $\mathcal{A}$. If $f \in C(\sigma(x))$ then
\begin{align*}
\Phi(f(x)) = f(\Phi(x))
\end{align*}

$\,$

{\bf \emph{Proof:}} The identity elements of $\mathcal{A}$ and $\mathcal{B}$ will be both denoted by $e$ and it will be clear from the context which one we are referring to.

First, we need to check that $f(\Phi(x))$ is a well-defined element of $\mathcal{B}$, i.e. that $\sigma(\Phi(x)) \subseteq \sigma(x)$. This is clear since, if $x - \lambda e$ is invertible for some $\lambda \in \mathbb{C}$, then $\Phi(x)-\lambda e = \Phi(x- \lambda e)$ is also invertible.

Let $\{p_n\}$ be sequence of polynomials in $C(\sigma(x))$ converging uniformly to $f$. Then we have that
\begin{itemize}
\item $\Phi(p_n(x)) \longrightarrow \Phi(f(x))$, by the continuity of $\Phi$ (see \PMlinkname{this entry}{HomomorphismsOfCAlgebrasAreContinuous}) and the continuity of the continuous functional calculus mapping.
\end{itemize}
\begin{itemize}
\item $p_n(\Phi(x)) \longrightarrow f(\Phi(x))$, by the continuity of the continuous functional calculus mapping.
\end{itemize}

It is easily checked that $\Phi(p_n(x)) = p_n(\Phi(x))$ (since $\Phi$ is an homomorphism). Hence we conclude that $\Phi(f(x)) = f(\Phi(x))$ as intended. $\square$
%%%%%
%%%%%
\end{document}
