\documentclass[12pt]{article}
\usepackage{pmmeta}
\pmcanonicalname{PolarDecomposition}
\pmcreated{2013-03-22 16:01:54}
\pmmodified{2013-03-22 16:01:54}
\pmowner{aube}{13953}
\pmmodifier{aube}{13953}
\pmtitle{polar decomposition}
\pmrecord{10}{38075}
\pmprivacy{1}
\pmauthor{aube}{13953}
\pmtype{Definition}
\pmcomment{trigger rebuild}
\pmclassification{msc}{47A05}

\endmetadata

% this is the default PlanetMath preamble.  as your knowledge
% of TeX increases, you will probably want to edit this, but
% it should be fine as is for beginners.

% almost certainly you want these
\usepackage{amssymb}
\usepackage{amsmath}
\usepackage{amsfonts}

\usepackage{mathrsfs}

\usepackage{amssymb, amsfonts,amsthm,amsxtra,amsmath}
\usepackage{latexsym}


% used for TeXing text within eps files
%\usepackage{psfrag}
% need this for including graphics (\includegraphics)
%\usepackage{graphicx}
% for neatly defining theorems and propositions
%\usepackage{amsthm}
% making logically defined graphics
%%%\usepackage{xypic}

% there are many more packages, add them here as you need them

% define commands here

\begin{document}
The polar decomposition of an operator is a generalization of the familiar factorization of a complex number $z$ in a radial part $|z|$ and an angular part $z/|z|$.

Let $\mathscr{H}$ be a Hilbert space, $x$ a bounded operator on $\mathscr{H}$. Then there exist a pair $(h,u)$, with $h$ a bounded positive operator and $u$ a partial isometry on $\mathscr{H}$, such that \[x=uh.\]\\ 

If we impose the further conditions that $1-u^*u$ is the projection to the kernel of $x$, and $\ker(h)=\ker(x)$, then $(h,u)$ is unique, and is called the \emph{polar decomposition} of $x$. The operator $h$ will be $|x|$, the square root of $x^*x$, and $u$ will be the partial isometry, determined by \begin{itemize} \item $u\xi=0$ for $\xi \in \ker(x)$ \item $u(|x|\xi)=x\xi$ for $\xi\in \mathscr{H}$. \end{itemize}

If $x$ is a closed, densely defined unbounded operator on $\mathscr{H}$, the polar decomposition $(u,h)$ still exists, where now $h$ will be the unbounded positive operator $|x|$ with the same domain $\mathscr{D}(x)$ as $x$, and $u$ still the partial isometry determined by \begin{itemize} \item $u\xi=0$ for $\xi \in \ker(x)$ \item $u(|x|\xi)=x\xi$ for $\xi\in \mathscr{D}(x)$. \end{itemize}

If $x$ is affiliated with a von Neumann algebra $M$, both $u$ and $h$ will be affiliated with $M$.
%%%%%
%%%%%
\end{document}
