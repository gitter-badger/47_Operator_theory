\documentclass[12pt]{article}
\usepackage{pmmeta}
\pmcanonicalname{SpectralTheorem1}
\pmcreated{2013-03-22 18:04:30}
\pmmodified{2013-03-22 18:04:30}
\pmowner{asteroid}{17536}
\pmmodifier{asteroid}{17536}
\pmtitle{spectral theorem}
\pmrecord{12}{40609}
\pmprivacy{1}
\pmauthor{asteroid}{17536}
\pmtype{Feature}
\pmcomment{trigger rebuild}
\pmclassification{msc}{47B15}
\pmclassification{msc}{47A15}
\pmclassification{msc}{47A10}
\pmclassification{msc}{46C99}
\pmclassification{msc}{15A18}

% this is the default PlanetMath preamble.  as your knowledge
% of TeX increases, you will probably want to edit this, but
% it should be fine as is for beginners.

% almost certainly you want these
\usepackage{amssymb}
\usepackage{amsmath}
\usepackage{amsfonts}

% used for TeXing text within eps files
%\usepackage{psfrag}
% need this for including graphics (\includegraphics)
%\usepackage{graphicx}
% for neatly defining theorems and propositions
%\usepackage{amsthm}
% making logically defined graphics
%%%\usepackage{xypic}

% there are many more packages, add them here as you need them

% define commands here

\begin{document}
\PMlinkescapephrase{theory}
\PMlinkescapephrase{state}
\PMlinkescapephrase{theory}
\PMlinkescapephrase{series}
\PMlinkescapephrase{bounded}
\PMlinkescapephrase{unbounded}

\section{Introduction}

The spectral theorem is series of results in functional analysis that explore conditions for operators in Hilbert spaces to be diagonalizable (in some appropriate sense). These results can also describe how the diagonalization takes \PMlinkescapetext{place}, mainly by analyzing how the operator acts in the underlying Hilbert space.

Roughly speaking, the spectral theorems state that normal operators (or self-adjoint operators) are diagonalizable and can be expressed as a sum or, more generally, as an integral of projections. More specifically, a normal (or self-adjoint) operator $T$ is unitarily equivalent to a multiplication operator in some \PMlinkname{$L^2$-space}{L2SpacesAreHilbertSpaces} and we can \PMlinkescapetext{associate} to it a spectral measure (or a resolution of identity) whose integration gives is $T$.

There is a wide \PMlinkescapetext{variety} of spectral theorems, each one with its own \PMlinkescapetext{specifications}, applicable to many classes of normal and self-adjoint operators.

\section{Motivation}

We explore here two \PMlinkescapetext{ways} to motivate the spectral theorem. The first is by recalling the finite-dimensional case, corresponding to the well known result in linear algebra, the \PMlinkname{spectral theorem for Hermitian matrices}{SpectralTheoremForHermitianMatrices} (or the self-adjoint analog). The second motivation comes from the \PMlinkname{$C^*$-algebra}{CAlgebra} theory, by regarding a normal (or self-adjoint) operator as a continuous function.

\subsubsection{Finite-dimensional case}



Suppose $T$ is a self-adjoint operator in a finite-dimensional Hilbert space $H$. An important fact about self-adjoint operators (not just in finite-dimensional spaces) is the following:

{\bf Fact 1 -} \emph{If $V \subseteq H$ is an invariant subspace by $T$, then so it is $V^{\perp}$, the orthogonal complement of $V$.}

{\bf \emph{Proof:}} Let $x \in V$ and $y \in V^{\perp}$. Then $\langle x, Ty\rangle = \langle Tx, y \rangle = 0$, where the last term is zero because $V$ is invariant by $T$, i.e. $Tx \in V$. But this proves that $Ty \in V^{\perp}$. $\square$ 

In finite dimensions it is known that every linear transformation has, at least, one eigenvector. Of course, the subspace generated by an eigenvector is always invariant.

Let $v_1$ be an eigenvector of $T$ and $V_1$ the subspace generated by it. For self-adjoint transformations, Fact 1 above says that $V_1^{\perp}$ is also invariant by $T$. Thus, by \PMlinkname{restriction}{RestrictionOfAFunction}, we have a self-adjoint operator $T: V_1^{\perp} \longrightarrow V_1^{\perp}$ and we could again find an eigenvector and repeat the same \PMlinkescapetext{argument}. Thus, we are decomposing $H$ as a direct sum of orthogonal one-dimensional subspaces $H = V_1 \oplus \dots \oplus V_n$, and the operator $T$ can be expressed as a sum
\begin{displaymath}
T = \sum_{i =1}^n \lambda_i P_i
\end{displaymath}
where each $\lambda_i$ is the eigenvalue associated with the eigenvector $v_i$ and each $P_i$ is the orthogonal projection onto the subspace $V_i$.

This is exactly the process of diagonalization of a self-adjoint matrix.

For normal operators it is more subtle as Fact 1 is no longer true. The idea to overpass this is that eigenvectors of normal operators are always orthogonal to each other (see \PMlinkname{this entry}{EigenvaluesOfNormalOperators}).

\subsubsection{$C^*$-algebras}

Suppose $T$ is a self-adjoint operator in some Hilbert space $H$. The closed *-algebra generated by $T$ and the identity operator is a commutative \PMlinkname{$C^*$-algebra}{CAlgebra}, which we denote by $C^*(T)$. Hence, the \PMlinkname{Gelfand-Naimark theorem}{GelfandTransform} and the continuous functional calculus provide an isomorphism

\begin{align*}
C^*(T) \cong C(\sigma(T))
\end{align*}
where $\sigma(T)$ stands for the spectrum of $T$ and $C(\sigma(T))$ is the $C^*$-algebra of continuous functions $\sigma(T) \to \mathbb{C}$.

Recall that the spectrum of a self-adjoint operator is a always a compact subset of $\mathbb{R}$. Thus, we can think of $T$ as a continuous function defined in a subset of $\mathbb{R}$.

It is a well known fact from measure theory that every continuous function $f:X \longrightarrow \mathbb{C}$ can be approximated by linear combinations of characteristic functions. With some additional effort it can be seen that each continuous function $f$ is in fact a (vector valued) integral of characteristic functions
\begin{align*}
\displaystyle f = \int_X f\, d\chi
\end{align*}
where $\chi$ is the vector measure of characteristic functions $\chi(A):= \chi_A$.

We now notice that characteristic functions in $\sigma(T)$ are not continuous in general. Hence, they may not have a correspondent in the $C^*(T)$, the $C^*$-algebra generated by $T$ and the identity. The \PMlinkescapetext{key} fact is that they do have a correspondent in the von Neumann algebra generated by $T$. Informally, this is the same as saying that characteristic functions belong to $L^{\infty}(\sigma(T))$ rather then $C(\sigma(T))$.

The correspondent operators in the von Neumann algebra generated by $T$ must be projections (since characteristic functions are projections in $L^{\infty}$), and similarly, $T$ can be approximated by linear combinations of projections and can, in fact, be expressed as an integral of projections:
\begin{align*}
T= \int_{\sigma(T)} \lambda \;dP(\lambda)
\end{align*}
where $P(\lambda)$ is a resolution of identity for $T$ (or a projection valued measure, when $T$ is a normal operator).

\section{Spectral Theorems}

Here we list a series of spectral theorems, applicable to different classes of normal or self-adjoint operators.

\subsubsection{Normal Operators}

\begin{itemize}
\item \PMlinkname{spectral theorem for Hermitean matrices}{SpectralTheoremForHermitianMatrices}

This is the finite-dimensional case. It is often referred to as "the spectral theorem", especially in linear algebra.
\end{itemize}
\begin{itemize}
\item spectral theorem for \PMlinkname{bounded}{OperatorNorm} normal operators in separable Hilbert spaces
\end{itemize}
\begin{itemize}
\item spectral theorem for bounded normal operators in inseparable Hilbert spaces
\end{itemize}
\begin{itemize}
\item spectral theorem for \PMlinkname{compact}{CompactOperator} normal operators
\end{itemize}
\begin{itemize}
\item spectral theorem for unitary operators
\end{itemize}
\begin{itemize}
\item spectral theorem for  \PMlinkname{unbounded}{OperatorNorm} normal operators
\end{itemize}

\subsubsection{Self-adjoint Operators}

\begin{itemize}
\item spectral theorem for self-adjoint matrices
\end{itemize}
\begin{itemize}
\item spectral theorem for bounded self-adjoint operators in separable Hilbert spaces
\end{itemize}
\begin{itemize}
\item spectral theorem for bounded self-adjoint operators in inseparable Hilbert spaces
\end{itemize}
\begin{itemize}
\item spectral theorem for \PMlinkescapetext{compact} self-adjoint operators
\end{itemize}
\begin{itemize}
\item spectral theorem for unbounded self-adjoint operators
\end{itemize}
%%%%%
%%%%%
\end{document}
