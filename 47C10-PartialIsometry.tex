\documentclass[12pt]{article}
\usepackage{pmmeta}
\pmcanonicalname{PartialIsometry}
\pmcreated{2013-03-22 15:50:50}
\pmmodified{2013-03-22 15:50:50}
\pmowner{CWoo}{3771}
\pmmodifier{CWoo}{3771}
\pmtitle{partial isometry}
\pmrecord{7}{37830}
\pmprivacy{1}
\pmauthor{CWoo}{3771}
\pmtype{Definition}
\pmcomment{trigger rebuild}
\pmclassification{msc}{47C10}
\pmdefines{unitary element}
\pmdefines{initial projection}
\pmdefines{final projection}

\usepackage{amssymb,amscd}
\usepackage{amsmath}
\usepackage{amsfonts}

% used for TeXing text within eps files
%\usepackage{psfrag}
% need this for including graphics (\includegraphics)
%\usepackage{graphicx}
% for neatly defining theorems and propositions
%\usepackage{amsthm}
% making logically defined graphics
%%%\usepackage{xypic}

% define commands here
\begin{document}
Partial isometry is a generalization of an isometry.  Before defining what a partial isometry is, let's recall two familiar concepts in linear algebra: an isometry and the adjoint of a linear map.

\begin{enumerate}
\item An isometry $T$ is a linear automorphism over an inner product space $V$ which preserves the inner product of any two vectors: $\langle x,y\rangle = \langle Tx, Ty\rangle$.  
\item The adjoint $T^*$ of a linear transformation $T$ is linear transformation such that $\langle Tx,y\rangle = \langle x, T^*y\rangle$, for any pair of vectors $x,y\in V$.

\end{enumerate}
If $V$ is non-singular with respect to the inner product $\langle \cdot,\cdot \rangle$ and that the adjoint $T^*$ of a linear transformation $T$ exists, it is not hard to show that 
\begin{quote}
$T$ is an isometry if and only if $TT^*=I=T^*T$.
\end{quote}
In other words, $T^*$ is the inverse of $T$.

More generally, in a ring with involution $*$, an isometry (or an unitary element) is a unit (both a left unit and a right unit) $a$ whose product with its adjoint $a^*$ is 1 (i.e. its inverse is its adjoint).  Now, if $a$ is not a unit, this product $aa^*$ 
will not be 1.  The next best thing to hope for is that the product will be an idempotent.  But because $aa^*$ is self-adjoint, this idempotent is in fact a projection.  This is how a partial isometry is defined.  Formally, 
\begin{quote}
let $R$ be a ring with involution $*$, an element $a\in R$ is a \emph{partial isometry} if $aa^*$ and $a^*a$ are both projections.
\end{quote}

Given a partial isometry $a$, the projections $a^*a$ and $aa^*$ are respectively called the \emph{initial projection} and \emph{final projection} of $a$.

\textbf{Examples}.  Under this definition, $0$ is a partial isometry, and so is any isometry.  

This definition can be readily applied to specific (more familiar) situations.  For example, if the ring in question is the ring of linear endomorphisms over a Euclidean space (real or complex), then a partial isometry is just a map such that its restriction to the complementary subspace of its kernel is an isometry.  If we look at the case when the space is 3 dimensional over the reals, and taking the standard basis, the matrix

\begin{center}$A =
\begin{pmatrix}
0 & 0 & 0 \\
0 & \sin\theta & -\cos\theta \\
0 & \cos\theta & \sin\theta
\end{pmatrix}$
\end{center}

corresponds to a partial isometry whose kernel is a line $L$.  Its restriction to the complement of $L$ corresponds to the matrix

\begin{center}$B =
\begin{pmatrix}
\sin\theta & -\cos\theta \\
\cos\theta & \sin\theta
\end{pmatrix}$,
\end{center}

which is an isometry (rotation).

\textbf{Remark}.  If the ring $R$ is a Baer *-ring, an element $a$ is a partial isometry iff $aa^*a=a$ (so $a^*aa^*=a^*$; $a$ and $a^*$ are generalized inverses of one another).
%%%%%
%%%%%
\end{document}
