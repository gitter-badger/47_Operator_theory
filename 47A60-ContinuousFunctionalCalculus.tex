\documentclass[12pt]{article}
\usepackage{pmmeta}
\pmcanonicalname{ContinuousFunctionalCalculus}
\pmcreated{2013-03-22 17:30:02}
\pmmodified{2013-03-22 17:30:02}
\pmowner{asteroid}{17536}
\pmmodifier{asteroid}{17536}
\pmtitle{continuous functional calculus}
\pmrecord{11}{39890}
\pmprivacy{1}
\pmauthor{asteroid}{17536}
\pmtype{Feature}
\pmcomment{trigger rebuild}
\pmclassification{msc}{47A60}
\pmclassification{msc}{46L05}
\pmclassification{msc}{46H30}
\pmrelated{FunctionalCalculus}
\pmrelated{PolynomialFunctionalCalculus}
\pmrelated{BorelFunctionalCalculus}
\pmdefines{continuous functions of normal operators}

\endmetadata

% this is the default PlanetMath preamble.  as your knowledge
% of TeX increases, you will probably want to edit this, but
% it should be fine as is for beginners.

% almost certainly you want these
\usepackage{amssymb}
\usepackage{amsmath}
\usepackage{amsfonts}

% used for TeXing text within eps files
%\usepackage{psfrag}
% need this for including graphics (\includegraphics)
%\usepackage{graphicx}
% for neatly defining theorems and propositions
%\usepackage{amsthm}
% making logically defined graphics
%%%\usepackage{xypic}

% there are many more packages, add them here as you need them

% define commands here

\begin{document}
\PMlinkescapeword{Lemma}

Let $B(H)$ be the algebra of bounded operators over a complex Hilbert space $H$. Let $T \in B(H)$ be a normal operator.

The {\bf continuous functional calculus} is a functional calculus which enables the expression
\begin{displaymath}
f(T)
\end{displaymath}
to make sense as a bounded operator in $H$, for continuous functions $f$.

More generally, when $\mathcal{A}$ is a \PMlinkname{$C^*$-algebra}{CAlgebra} with identity element $e$, and $x$ is a normal element of $\mathcal{A}$, the \emph{continuous functional calculus} allows one to define $f(x) \in \mathcal{A}$ when $f$ is a continuous function.

More precisely, if $\sigma(x)$ denotes the spectrum of $x$ and $C(\sigma(x))$ denotes the $C^*$-algebra of complex valued continuous functions on $\sigma(x)$, we will define a continuous homomorphism 
\begin{center}
$C(\sigma(x)) \longrightarrow \mathcal{A}$

$f \mapsto f(x)$
\end{center}
that \PMlinkescapetext{satisfies} the \PMlinkname{functional calculus}{FunctionalCalculus} \PMlinkescapetext{properties}.

There are several reasons to require the continuity of $f$ on the spectrum $\sigma(x)$.

For example, suppose $\lambda_0 \in \sigma(x)$. The function $f(\lambda) = \frac{1}{\lambda - \lambda_0}$ is clearly not continuous in $\lambda_0$. By the functional calculus \PMlinkescapetext{properties} we would obtain
\begin{displaymath}
f(x)= \frac{1}{x-\lambda_0 e} = (x - \lambda_0 e)^{-1}
\end{displaymath}

but $x - \lambda_0 e$ is not invertible since $\lambda_0 \in \sigma(x)$.

The abstraction towards $C^*$-algebras is almost \PMlinkescapetext{necessary}. Indeed, $C^*$-algebras are the appropriate \PMlinkescapetext{object} where to \PMlinkescapetext{state} and prove the continuous functional calculus. The conclusions towards $B(H)$ then follow as a particular case.

\section{Preliminary construction}

Let $\mathcal{A}$ be a unital $C^*$-algebra and $x$ a normal element in $\mathcal{A}$. Let $\mathcal{B} \subseteq \mathcal{A}$ be the $C^*$-subalgebra generated by $x$ and the identity $e$ of $\mathcal{A}$.

Thus, $\mathcal{B}$ is the norm closure of the algebra generated by $x$, $x^*$ and $e$.

Moreover, since $x$ is \PMlinkescapetext{normal}, $x^*x=xx^*$, it follows that $\mathcal{B}$ is commutative and $\mathcal{B}$ consists of those elements $y \in \mathcal{A}$ that can be approximated by polynomials $p(x,x^*)$ in $x$ and $x^*$.

Recall the following facts:
\begin{itemize}
\item Since $\mathcal{B}$ is a commutative unital $C^*$-algebra, the set $\bigtriangleup$ of multiplicative linear functionals on $\mathcal{B}$ is a compact Hausdorff space.
 
\item Let $C(\bigtriangleup)$ denote the $C^*$-algebra of complex valued continuous functions on $\bigtriangleup$. The Gelfand transform $G : \mathcal{B} \longrightarrow C(\bigtriangleup)$ is a $C^*$-isomorphism.
\end{itemize}

The following result is perhaps the \PMlinkescapetext{key} for the definition of the continuous functional calculus.

$\,$

{\bf Theorem 1 -} \emph{$\bigtriangleup$ and $\sigma(x)$ are homeomorphic topological spaces. Moreover, the mapping $S: \bigtriangleup \to \sigma(x)$ defined by}
\begin{align*}
S(\phi):= \phi(x)
\end{align*}
\emph{is such an homeomorphism.}


{\bf \emph{\PMlinkescapetext{Proof} :}} We need to check that $S$ is well defined, i.e. $\phi(x) \in \sigma(x)$ for all $\phi \in \bigtriangleup$.

From the identity
\begin{displaymath}
\phi(x-\phi(x)e) = \phi(x)-\phi(x)\phi(e) = \phi(x)-\phi(x)=0
\end{displaymath}
follows that $x-\phi(x)e$ cannot be invertible in $\mathcal{B}$ (recall that $\phi$ is a multiplicative linear functional on $\mathcal{B}$).

Thus, $\phi(x) \in \sigma_{\mathcal{B}}(x)$. By the spectral invariance theorem, we see that $\phi(x) \in \sigma(x) = \sigma_{\mathcal{B}}(x)$, and so $S$ is well defined.

\begin{itemize}
\item {\bf $S$ is continuous -} Suppose $\phi_{\alpha}$ is a net in $\bigtriangleup$ such that $\phi_{\alpha} \longrightarrow \phi$. Recall that the topology in $\bigtriangleup$ is the weak-* topology, so $\phi_{\alpha}(x) \longrightarrow \phi(x)$.

Thus, $S(\phi_{\alpha}) \longrightarrow S(\phi)$ and so $S$ is continuous.

\item {\bf $S$ is \PMlinkescapetext{injective} -} Suppose $S(\phi_1) = S(\phi_2)$. Then, $ \phi_1(x) = \phi_2(x)$. Since 
\begin{displaymath}
\phi_i(x^*)=G(x^*)(\phi_i) = \overline{G(x)(\phi_i)}=\overline{\phi_i(x)}
\end{displaymath}
we must also have $\phi_1(x^*)=\phi_2(x^*)$.

This clearly implies that
\begin{displaymath}
\phi_1(p(x,x^*))= \phi_2(p(x,x^*))
\end{displaymath}
for every polynomial in two variables $p$.

Recall that the "polynomials" $p(x,x^*)$ are dense in $\mathcal{B}$. So we must have $\phi_1(y)=\phi_2(y)$ for every $y \in \mathcal{B}$, i.e. $\phi_1 = \phi_2$.

\item {\bf $S$ is \PMlinkescapetext{surjective} -} Let $\lambda \in \sigma(x) = \sigma_{\mathcal{B}}(x)$. Then $x - \lambda e$ is not invertible.

Since $\mathcal{B}$ is commutative, $x - \lambda e$ is contained in a maximal ideal $\mathcal{M}$.

As $\mathcal{M}$ is maximal ideal, the quotient $\mathcal{B}/\mathcal{M}$ is a division algebra, and so by the Gelfand-Mazur theorem, $\mathcal{B}/\mathcal{M}$ must ne isomorphic to $\mathbb{C}$.

Therefore the quotient homomorphism
\begin{displaymath}
\phi : \mathcal{B} \longrightarrow \mathcal{B}/\mathcal{M}=\mathbb{C}
\end{displaymath}
is a multiplicative linear functional such that $\phi(x-\lambda e) = 0$, i.e. $\phi(x) = \lambda$, i.e. $S(\phi) = \lambda$.

Therefore, $S$ is surjective.

\end{itemize}

Since $S$ is a continuous bijective function from the compact Hausdorff space $\bigtriangleup$ to $\sigma(x)$, it follows that it must be a homeomorphism. $\square$

\section{Definition of the continuous functional calculus}

If $S$ is the homeomorphism between $\bigtriangleup$ and $\sigma(x)$ defined as above, then the mapping $f \mapsto f \circ S$ is a *-isomorphism between the algebras $C(\sigma(x))$ and $C(\bigtriangleup)$. Since the Gelfand transform $G : \mathcal{B} \longrightarrow C(\bigtriangleup)$ is a also a *-isomorphism, we obtain a *-isomorphism
\begin{displaymath}
\Gamma : C(\sigma(x)) \longrightarrow \mathcal{B}
\end{displaymath}
by setting $\Gamma (f):= G^{-1}(f \circ S)$.

{\bf Definition -} \emph{Suppose $x$ is a normal element in a unital $C^*$-algebra $\mathcal{A}$. For every $f \in C(\sigma(x))$ we define}
\begin{displaymath}
f(x):=\Gamma(f)\;\; \in \mathcal{B} \subseteq \mathcal{A}
\end{displaymath}

\emph{The mapping $\Gamma$, such that $f \mapsto f(x)$, is called the} {\bf continuous functional calculus} \emph{for $x$.}

We now prove the \PMlinkname{functional calculus}{FunctionalCalculus} \PMlinkescapetext{properties} for the continuous functional calculus and show its uniqueness:

$\,$

{\bf Theorem 2 -} \emph{Let $\mathcal{A}$ be a unital $C^*$-algebra, $x \in \mathcal{A}$ a normal element and $\mathrm{id}$ the identity function in $\mathbb{C}$. The continuous functional calculus for $x$ is the unique unital *-homomorphism between $C(\sigma(x))$ and $\mathcal{A}$ which sends $\mathrm{id}$ to $x$. In particular, for every polynomial $p$ in $\mathbb{C}$ of the form $p(\lambda):= \sum c_{n,m}\, \lambda^n \overline{\lambda}^m$, we have $p(x) = \sum c_{n,m}\, x^n (x^*)^m$.}

{\bf \emph{\PMlinkescapetext{Proof} :}} We have seen that the continuous functional calculus $\Gamma$ for $x$ is a *-homomorphism between $C(\sigma(x))$ and $\mathcal{A}$. Recall that $\Gamma$ was defined by $\Gamma (f) := G^{-1} (f \circ S)$. It is clear by the definition that $\Gamma$ is unital. Also, $G \circ \Gamma (f) = f \circ S$ for every $f \in C(\bigtriangleup)$. Taking the identity function $\mathrm{id}$ we obtain that for every $\phi \in \bigtriangleup$
\begin{eqnarray*}
G \circ \Gamma (\mathrm{id}) (\phi) & = & \mathrm{id} (S(\phi)) \\
& = & \phi(x) \\
& = & G(x) (\phi)
\end{eqnarray*}
Since the Gelfand transform is a *-isomorphism, we must have $\Gamma(\mathrm{id}) = x$. 

Now, let $p: \mathbb{C} \to \mathbb{C}$ be a polynomial of the form $p(\lambda):= \sum c_{n,m}\, \lambda^n \overline{\lambda}^m$. Notice that $p = \sum c_{n,m}\, \mathrm{id}^n \overline{\mathrm{id}}^m$. If $F$ is any unital *-homomorphism such that $F(\mathrm{id})= x$, then one must have $F(p) = \sum c_{n,m}\, x^n(x^*)^m$. Thus all such unital *-homomorphisms coincide on the subspace of polynomials of the above form. By the \PMlinkname{Stone-Weierstrass theorem}{StoneWeierstrassTheoremComplexVersion}, this subspace is dense in $C(\sigma(x))$. Thus, all such unital *-homomorphisms coincide in $C(\sigma(x))$, and uniqueness is proven. $\square$

\section{Properties}
\begin{itemize}
\item The spectral mapping theorem assures that for $f \in C(\sigma(x))$
\begin{displaymath}
\sigma(f(x)) = f(\sigma(x))
\end{displaymath}

\item When $\sigma(x) \subset \mathbb{R}^+$ the continuous functional calculus assures the existence of a square root $\sqrt{x}$ of $x$, since $\sqrt{\lambda}$ is defined and continuous on $\lambda \in \sigma(x)$. 
\end{itemize}
%%%%%
%%%%%
\end{document}
