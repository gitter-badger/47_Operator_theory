\documentclass[12pt]{article}
\usepackage{pmmeta}
\pmcanonicalname{ResolventMatrix}
\pmcreated{2013-03-22 13:36:20}
\pmmodified{2013-03-22 13:36:20}
\pmowner{mps}{409}
\pmmodifier{mps}{409}
\pmtitle{resolvent matrix}
\pmrecord{8}{34235}
\pmprivacy{1}
\pmauthor{mps}{409}
\pmtype{Definition}
\pmcomment{trigger rebuild}
\pmclassification{msc}{47A10}
\pmclassification{msc}{15A15}
\pmdefines{resolvent}

% this is the default PlanetMath preamble.  as your knowledge
% of TeX increases, you will probably want to edit this, but
% it should be fine as is for beginners.

% almost certainly you want these
\usepackage{amssymb}
\usepackage{amsmath}
\usepackage{amsfonts}

% used for TeXing text within eps files
%\usepackage{psfrag}
% need this for including graphics (\includegraphics)
%\usepackage{graphicx}
% for neatly defining theorems and propositions
%\usepackage{amsthm}
% making logically defined graphics
%%%\usepackage{xypic}

% there are many more packages, add them here as you need them

% define commands here
\begin{document}
The \emph{resolvent matrix} of a matrix $A$ is defined as
\[
R_{A}(s)=(sI-A)^{-1}.
\]

Note: $I$ is the identity matrix and $s$ is a complex variable. Also note that $R_{A}(s)$ is undefined on $Sp(A)$ (the spectrum of $A$).

More generally, let $A$ be a unital algebra over the field of complex numbers $\mathbb{C}$.  The \emph{resolvent} $R_x$ of an element $x\in A$ is a function from $\mathbb{C}-Sp(x)$ to $A$ given by
\[
R_x(s)=(s\cdot 1-x)^{-1}
\]
where $Sp(x)$ is the spectrum of $x$: $Sp(x)=\lbrace t\in \mathbb{C}\mid t\cdot 1 -x\mbox{ is not invertible in }A\rbrace$.

If $A$ is commutative and $s\notin Sp(x)\cup Sp(y)$, then $R_x(s)-R_y(s)=R_x(s)R_y(s)(x-y)$.
%%%%%
%%%%%
\end{document}
