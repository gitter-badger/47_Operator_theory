\documentclass[12pt]{article}
\usepackage{pmmeta}
\pmcanonicalname{ProofOfTaylorsFormulaForMatrixFunctions}
\pmcreated{2013-03-22 17:57:04}
\pmmodified{2013-03-22 17:57:04}
\pmowner{joen235}{18354}
\pmmodifier{joen235}{18354}
\pmtitle{proof of Taylor's formula for matrix functions}
\pmrecord{4}{40450}
\pmprivacy{1}
\pmauthor{joen235}{18354}
\pmtype{Proof}
\pmcomment{trigger rebuild}
\pmclassification{msc}{47A56}

\endmetadata

% this is the default PlanetMath preamble.  as your knowledge
% of TeX increases, you will probably want to edit this, but
% it should be fine as is for beginners.

% almost certainly you want these
\usepackage{amssymb}
\usepackage{amsmath}
\usepackage{amsfonts}

% used for TeXing text within eps files
%\usepackage{psfrag}
% need this for including graphics (\includegraphics)
%\usepackage{graphicx}
% for neatly defining theorems and propositions
\usepackage{amsthm}
% making logically defined graphics
%%%\usepackage{xypic}

% there are many more packages, add them here as you need them

% define commands here
 \theoremstyle{plain}
  \newtheorem*{thm*}{Theorem}
\begin{document}
\begin{thm*}
Let $p$ be a polynomial and suppose $\mathbf{A}$ and $\mathbf{B}$
are squared matrices of the same size, then ${\displaystyle
p(\mathbf{A}+\mathbf{B})=\sum_{k=0}^{n}\frac{1}{k!}p^{(k)}(\mathbf{A})\mathbf{B}^{k}}$
where $n=\deg(p)$.
\end{thm*}

\begin{proof}
Since $p$ is a polynomial, we can apply the Taylor expansion:

$$p(x)=\sum_{k=0}^{n}\frac{1}{k!}p^{\left(k\right)}\left(x_{0}\right)\left(x-x_{0}\right)^{k}$$
where $ n=\deg(p)$. Now let $x=\mathbf{A}+\mathbf{B}$ and
$x_0=\mathbf{A}$.

The Taylor expansion can be checked as follows: let
$p\left(x\right)=\sum_{k=0}^{n}a_{k}x^{k}$ for coefficients $a_{k}$
(note that this coefficients can be taken from the space of square
matrices defined over a field). We define the formal derivative of
this polynomial as $p^{\left(1\right)}\left(x\right) =
\frac{dp}{dx}=\sum_{k=1}^{n}a_{k}kx^{k-1}$ and we define
$p^{\left(k\right)}=\frac{dp^{\left(k-1\right)}}{dx}$.

 Then
$p^{\left(k\right)}\left(x\right)=\sum_{i=k}^{n}a_{i}\frac{i!}{\left(i-k\right)!}x^{i-k}$
and we have
$\frac{1}{k!}p^{\left(k\right)}\left(x_{0}\right)=\sum_{i=k}^{n}a_{i}\frac{i!}{\left(i-k\right)!k!}\left(x_{0}\right)^{i-k}$.
Now consider{\small
\begin{eqnarray*}
 &  & \sum_{k=0}^{n}\frac{1}{k!}p^{\left(k\right)}\left(x_{0}\right)\left(x-x_{0}\right)^{k}=\sum_{k=0}^{n}\left(\sum_{i=k}^{n}a_{i}\frac{i!}{\left(i-k\right)!k!}\left(x_{0}\right)^{i-k}\left(x-x_{0}\right)^{k}\right)\\
 & = & \sum_{i=0}^{n}a_{i}\left(x_{0}\right)^{i}+\sum_{i=1}^{n}a_{i}i\left(x_{0}\right)^{i-1}\left(x-x_{0}\right)+\dots+\sum_{i=j}^{n}a_{i}\frac{i!}{\left(i-j\right)!j!}\left(x_{0}\right)^{i-j}\left(x-x_{0}\right)^{j}+\dots+a_{n}\left(x-x_{0}\right)^{n}\\
 & = & a_{0}+a_{1}\left(x\right)+\dots+a_{i}\left(\sum_{j=0}^{i}\frac{i!}{\left(i-j\right)!j!}\left(x_{0}\right)^{i-j}\left(x-x_{0}\right)^{j}\right)+\dots+a_{n}\left(\sum_{j=0}^{n}\frac{n!}{\left(n-j\right)!j!}\left(x_{0}\right)^{n-j}\left(x-x_{0}\right)^{j}\right)\\
 & = & \sum_{k=0}^{n}a_{k}x^{i}=p\left(x\right)\end{eqnarray*}
 }
since
$\sum_{j=0}^{i}\frac{i!}{\left(i-j\right)!j!}\left(x_{0}\right)^{i-j}\left(x-x_{0}\right)^{j}=\left(x\right)^{i}$.
\end{proof}

%%%%%
%%%%%
\end{document}
