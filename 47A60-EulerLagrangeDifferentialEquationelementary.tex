\documentclass[12pt]{article}
\usepackage{pmmeta}
\pmcanonicalname{EulerLagrangeDifferentialEquationelementary}
\pmcreated{2013-03-22 12:21:49}
\pmmodified{2013-03-22 12:21:49}
\pmowner{rspuzio}{6075}
\pmmodifier{rspuzio}{6075}
\pmtitle{Euler-Lagrange differential equation (elementary)}
\pmrecord{34}{32092}
\pmprivacy{1}
\pmauthor{rspuzio}{6075}
\pmtype{Definition}
\pmcomment{trigger rebuild}
\pmclassification{msc}{47A60}
\pmclassification{msc}{70H03}
\pmclassification{msc}{49K05}
\pmsynonym{Euler-Lagrange condition}{EulerLagrangeDifferentialEquationelementary}
\pmrelated{CalculusOfVariations}
\pmrelated{BeltramiIdentity}
\pmrelated{VersionOfTheFundamentalLemmaOfCalculusOfVariations}
\pmdefines{Euler-Lagrange differential equation}
\pmdefines{Lagrangian}

\usepackage{amssymb}
\usepackage{amsmath}
\usepackage{amsfonts}

% used for TeXing text within eps files
%\usepackage{psfrag}
% need this for including graphics (\includegraphics)
%\usepackage{graphicx}
% for neatly defining theorems and propositions
%\usepackage{amsthm}
% making logically defined graphics
%%%\usepackage{xypic} 

% there are many more packages, add them here as you need them

% define commands here
\newcommand{\md}{d}
\newcommand{\mv}[1]{\mathbf{#1}}	% matrix or vector
\newcommand{\mvt}[1]{\mv{#1}^{\mathrm{T}}}
\newcommand{\mvi}[1]{\mv{#1}^{-1}}
\newcommand{\mderiv}[1]{\frac{\md}{\md {#1}}} %d/dx
\newcommand{\mnthderiv}[2]{\frac{\md^{#2}}{\md {#1}^{#2}}} %d^n/dx
\newcommand{\mpderiv}[1]{\frac{\partial}{\partial {#1}}} %partial d^n/dx
\newcommand{\mnthpderiv}[2]{\frac{\partial^{#2}}{\partial {#1}^{#2}}} %partial d^n/dx
\newcommand{\borel}{\mathfrak{B}}
\newcommand{\integers}{\mathbb{Z}}
\newcommand{\rationals}{\mathbb{Q}}
\newcommand{\reals}{\mathbb{R}}
\newcommand{\complexes}{\mathbb{C}}
\newcommand{\naturals}{\mathbb{N}}
\newcommand{\defined}{:=}
\newcommand{\var}{\mathrm{var}}
\newcommand{\cov}{\mathrm{cov}}
\newcommand{\corr}{\mathrm{corr}}
\newcommand{\set}[1]{\left\{#1\right\}}
\newcommand{\powerset}[1]{\mathcal{P}(#1)}
\newcommand{\bra}[1]{\langle#1 \vert}
\newcommand{\ket}[1]{\vert \hspace{1pt}#1\rangle}
\newcommand{\braket}[2]{\langle #1 \ket{#2}}
\newcommand{\abs}[1]{\left|#1\right|}
\newcommand{\norm}[1]{\left|\left|#1\right|\right|}
\newcommand{\esssup}{\mathrm{ess\ sup}}
\newcommand{\Lspace}[1]{L^{#1}}
\newcommand{\Lone}{\Lspace{1}}
\newcommand{\Ltwo}{\Lspace{2}}
\newcommand{\Lp}{\Lspace{p}}
\newcommand{\Lq}{\Lspace{q}}
\newcommand{\Linf}{\Lspace{\infty}}
\begin{document}
Let $q(t)$ be a twice differentiable function from $\reals$ to $\reals$ and let $L$ be a twice differentiable function from $\reals^3$ to $\reals$.  Let
$\dot{q}$ denote $\mderiv{t}{q}$.

Define the functional $I$ as follows:
 $$I(q) = \int_a^b L (t, q(t), \dot{q}(t)) \, dt$$
Suppose we regard the function $L$ and the limits of integratiuon $a$ and $b$
as fixed and allow $q$ to vary.  Then we could ask for which functions $q$ (if any) this integral attains an extremal (minimum or maximum) value.  (Note: especially in Physics literature, the function $L$ is known as the \emph{Lagrangian}.)

Suppose that a differentiable function \,$q_0\!: [a,b] \to \reals$\, is an extremum of $I$.  Then, for every differentiable function \,$f \colon [-1,+1] \times [a,b] \to \reals$\, such that\, $f(0,x) = q_0 (x)$, the function 
\,$g \colon [-1, +1] \to \reals$,\, defined as
 $$g (\lambda) = \int_a^b L \left( t, f(\lambda, t), {\partial f \over \partial t} (\lambda, t) \right) \, dt$$
will have an extremum at $\lambda = 0$.  If this function is differentiable, then\, $dg/d\lambda = 0$\, when \,$\lambda = 0$. 

By studying the condition $dg/d\lambda = 0$ (see the addendum to this entry for details), one sees that, if a function $q$ is to be an extremum of the integral $I$, then $q$ must satisfy the following equation:
\begin{equation}
\mpderiv{q} L - \mderiv{t} \left( \mpderiv{\dot{q}} L \right) = 0.
\end{equation}
This equation is known as the \emph{Euler--Lagrange differential equation} or  the Euler-Lagrange condition.  A few comments on notation might be in \PMlinkescapetext{order}.  The notations $\mpderiv{q} L$ and $\mpderiv{\dot{q}} L$ denote the partial derivatives of the function $L$ with respect to its second and third arguments, respectively.  The notation $\mderiv{t}$ means that one is to first make the argument a function of $t$ by replacing the second argument with $q(t)$ and the third argument with $\dot{q}(t)$ and secondly, differentiate the resulting function with respect to $t$.  Using the chain rule, the Euler-Lagrange equation can be written as follows:
\begin{equation}
\mpderiv{q} L - {\partial^2 \over \partial t \partial \dot{q}} L - \dot{q} {\partial^2 \over \partial q \partial \dot{q}} L - \ddot{q} {\partial^2 \over  \partial \dot{q}^2} L = 0
\end{equation}

This equation plays an important role in the calculus of variations.  In using this equation, it must be remembered that it is only a necessary condition and, hence, given a solution of this equation, one cannot \PMlinkescapetext{jump} to the conclusion that this solution is a local extremum of the functional $F$.  More work is needed to determine whether the solution of the Euler-Lagrange equation is an extremum of the integral $I$ or not.

In the special case $\mpderiv{t} L = 0$, the Euler-Lagrange equation can be replaced by the Beltrami identity.
%%%%%
%%%%%
\end{document}
