\documentclass[12pt]{article}
\usepackage{pmmeta}
\pmcanonicalname{FunctionalCalculusForHermitianMatrices}
\pmcreated{2013-03-22 14:40:12}
\pmmodified{2013-03-22 14:40:12}
\pmowner{mathcam}{2727}
\pmmodifier{mathcam}{2727}
\pmtitle{functional calculus for Hermitian matrices}
\pmrecord{4}{36271}
\pmprivacy{1}
\pmauthor{mathcam}{2727}
\pmtype{Definition}
\pmcomment{trigger rebuild}
\pmclassification{msc}{47C05}
\pmrelated{FunctionalCalculus}

% this is the default PlanetMath preamble.  as your knowledge
% of TeX increases, you will probably want to edit this, but
% it should be fine as is for beginners.

% almost certainly you want these
\usepackage{amssymb}
\usepackage{amsmath}
\usepackage{amsfonts}
\usepackage{amsthm}

% used for TeXing text within eps files
%\usepackage{psfrag}
% need this for including graphics (\includegraphics)
%\usepackage{graphicx}
% for neatly defining theorems and propositions
%\usepackage{amsthm}
% making logically defined graphics
%%%\usepackage{xypic}

% there are many more packages, add them here as you need them

% define commands here

\newcommand{\mc}{\mathcal}
\newcommand{\mb}{\mathbb}
\newcommand{\mf}{\mathfrak}
\newcommand{\ol}{\overline}
\newcommand{\ra}{\rightarrow}
\newcommand{\la}{\leftarrow}
\newcommand{\La}{\Leftarrow}
\newcommand{\Ra}{\Rightarrow}
\newcommand{\nor}{\vartriangleleft}
\newcommand{\Gal}{\text{Gal}}
\newcommand{\GL}{\text{GL}}
\newcommand{\Z}{\mb{Z}}
\newcommand{\R}{\mb{R}}
\newcommand{\Q}{\mb{Q}}
\newcommand{\C}{\mb{C}}
\newcommand{\<}{\langle}
\renewcommand{\>}{\rangle}
\begin{document}
Let $I\subset\mathbb{R}$ be a real interval, $f$ a real-valued function on $I$, and let $M$ be an $n\times n$ real symmetric (and thus Hermitian) matrix whose eigenvalues are contained in $I$.

By the spectral theorem, we can diagonalize $M$ by an orthogonal matrix $O$, so we can write $M=ODO^{-1}$ where $D$ is the diagonal matrix consisting of the eigenvalues $\{\lambda_1,\lambda_2,\ldots,\lambda_n\}$.  We then define

\begin{align*}
f(A)=Of(D)O^{-1},
\end{align*}

where $f(D)$ denotes the diagonal matrix whose diagonal entries are given by $f(\lambda_i)$.

It is easy to verify that $f(A)$ is well-defined, i.e. a permutation of the eigenvalues corresponds to the same definition of $f(A)$.
%%%%%
%%%%%
\end{document}
