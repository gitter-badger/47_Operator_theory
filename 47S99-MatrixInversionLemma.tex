\documentclass[12pt]{article}
\usepackage{pmmeta}
\pmcanonicalname{MatrixInversionLemma}
\pmcreated{2013-03-22 15:38:44}
\pmmodified{2013-03-22 15:38:44}
\pmowner{kronos}{12218}
\pmmodifier{kronos}{12218}
\pmtitle{matrix inversion lemma}
\pmrecord{6}{37577}
\pmprivacy{1}
\pmauthor{kronos}{12218}
\pmtype{Result}
\pmcomment{trigger rebuild}
\pmclassification{msc}{47S99}
\pmsynonym{Sherman-Morrison formula}{MatrixInversionLemma}
\pmsynonym{Woodbury matrix identity}{MatrixInversionLemma}
\pmsynonym{Woodbury formula}{MatrixInversionLemma}
\pmsynonym{rank-k correction}{MatrixInversionLemma}
\pmrelated{SchurComplement}

% this is the default PlanetMath preamble.  as your knowledge
% of TeX increases, you will probably want to edit this, but
% it should be fine as is for beginners.

% almost certainly you want these
\usepackage{amssymb}
\usepackage{amsmath}
\usepackage{amsfonts}

% used for TeXing text within eps files
%\usepackage{psfrag}
% need this for including graphics (\includegraphics)
%\usepackage{graphicx}
% for neatly defining theorems and propositions
%\usepackage{amsthm}
% making logically defined graphics
%%%\usepackage{xypic}

% there are many more packages, add them here as you need them

% define commands here
\begin{document}
These frequently used formulae allow to quickly calculate the inverse of a slight modification of an operator (matrix) $x$, given that $x^{-1}$ is already known.

The matrix inversion lemma states that $$\left( x+ s \sigma z^* \right)^{-1} = x^{-1} - x^{-1} s \left( \sigma^{-1} + z^* x^{-1} s \right)^{-1} z^* x^{-1},$$ where $x$, $s$, $z^*$ and $\sigma$ are operators (matrices) of appropriate size. The formula especially is convenient if the rank of the regular $\sigma$ is 1, or small in comparison to $x$'s rank.

This identity, involving the inverse of Schur's complement $d- z^* x^{-1} s$ (hopefully this may be easily computed) holds as well: $$\begin{bmatrix} x & s \\ z^* & d \end{bmatrix}^{-1} = \begin{bmatrix} x^{-1}+ x^{-1} s (d- z^* x^{-1} s)^{-1} z^* x^{-1} & -x^{-1} s (d-z^*x^{-1}s)^{-1} \\ -(d-z^*x^{-1}s)^{-1} z^* x^{-1} & (d-z^*x^{-1}s)^{-1} \end{bmatrix}.$$
%%%%%
%%%%%
\end{document}
