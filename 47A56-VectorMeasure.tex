\documentclass[12pt]{article}
\usepackage{pmmeta}
\pmcanonicalname{VectorMeasure}
\pmcreated{2013-03-22 17:29:23}
\pmmodified{2013-03-22 17:29:23}
\pmowner{asteroid}{17536}
\pmmodifier{asteroid}{17536}
\pmtitle{vector measure}
\pmrecord{12}{39877}
\pmprivacy{1}
\pmauthor{asteroid}{17536}
\pmtype{Definition}
\pmcomment{trigger rebuild}
\pmclassification{msc}{47A56}
\pmclassification{msc}{46G12}
\pmclassification{msc}{46G10}
\pmclassification{msc}{28C20}
\pmclassification{msc}{28B05}
\pmdefines{complex measure}
\pmdefines{countably additive vector measure}

% this is the default PlanetMath preamble.  as your knowledge
% of TeX increases, you will probably want to edit this, but
% it should be fine as is for beginners.

% almost certainly you want these
\usepackage{amssymb}
\usepackage{amsmath}
\usepackage{amsfonts}

% used for TeXing text within eps files
%\usepackage{psfrag}
% need this for including graphics (\includegraphics)
%\usepackage{graphicx}
% for neatly defining theorems and propositions
%\usepackage{amsthm}
% making logically defined graphics
%%%\usepackage{xypic}

% there are many more packages, add them here as you need them

% define commands here

\begin{document}
Let $S$ be a set and $\mathcal{F}$ a field of sets of $S$. Let $X$ be a topological vector space.

A {\bf vector measure} is a function $\mu : \mathcal{F} \longrightarrow X$ that is \PMlinkescapetext{finitely additive}, i.e. for any two disjoint sets $A_1, A_2$ in $\mathcal{F}$ we have
\begin{displaymath}
\mu(A_1 \cup A_2) = \mu(A_1)+\mu(A_2)
\end{displaymath}

A vector measure $\mu$ is said to be {\bf \PMlinkescapetext{countably additive}} if for any sequence $(A_n)_{n \in \mathbb{N}}$ of disjoint sets in $\mathcal{F}$ such that $\displaystyle \bigcup_{n =1}^{\infty}A_n \in \mathcal{F}$ one has
\begin{displaymath}
\mu(\bigcup_{n =1}^{\infty}A_n) = \sum_{n=1}^{\infty} \mu(A_n)
\end{displaymath}
where the series converges in the topology of $X$.

In the particular case when $X = \mathbb{C}$, a countably additive vector measure is usually called a {\bf complex measure}.

Thus, vector measures are \PMlinkescapetext{similar} to measures and signed measures but they take values on a vector space (with a particular topology).

\subsubsection{Examples :}
\begin{itemize}
\item Let $(X, \mathfrak{B}, \lambda)$ be a measure space. Consider the Banach space \PMlinkname{$L^p(X, \mathfrak{B}, \lambda)$}{LpSpace} with $1 \leq p \leq \infty$. Define the the function $\mu : \mathfrak{B} \longrightarrow L^p(X, \mathfrak{B}, \mu)$ by
\begin{displaymath}
\mu(A) := \chi_A
\end{displaymath}
where $\chi_A$ denotes the characteristic function of the measurable set $A$. It is easily seen that $\mu$ is a vector measure, which is countably additive if $1 \leq p < \infty$ (in case $p = \infty$, countably additiveness fails).
\end{itemize}
\begin{itemize}
\item spectral measures are vector measures in the $\sigma$-algebra of Borel sets in $\mathbb{C}$ whose values are projections on some Hilbert space. They are used in general formulations of the spectral theorem.
\end{itemize}
%%%%%
%%%%%
\end{document}
