\documentclass[12pt]{article}
\usepackage{pmmeta}
\pmcanonicalname{ExamplesOfBoundedAndUnboundedOperators}
\pmcreated{2013-03-22 15:17:37}
\pmmodified{2013-03-22 15:17:37}
\pmowner{matte}{1858}
\pmmodifier{matte}{1858}
\pmtitle{examples of bounded and unbounded operators}
\pmrecord{12}{37091}
\pmprivacy{1}
\pmauthor{matte}{1858}
\pmtype{Example}
\pmcomment{trigger rebuild}
\pmclassification{msc}{47L25}

\endmetadata

% this is the default PlanetMath preamble.  as your knowledge
% of TeX increases, you will probably want to edit this, but
% it should be fine as is for beginners.

% almost certainly you want these
\usepackage{amssymb}
\usepackage{amsmath}
\usepackage{amsfonts}
\usepackage{amsthm}

\usepackage{mathrsfs}

% used for TeXing text within eps files
%\usepackage{psfrag}
% need this for including graphics (\includegraphics)
%\usepackage{graphicx}
% for neatly defining theorems and propositions
%
% making logically defined graphics
%%%\usepackage{xypic}

% there are many more packages, add them here as you need them

% define commands here

\newcommand{\sR}[0]{\mathbb{R}}
\newcommand{\sC}[0]{\mathbb{C}}
\newcommand{\sN}[0]{\mathbb{N}}
\newcommand{\sZ}[0]{\mathbb{Z}}

 \usepackage{bbm}
 \newcommand{\Z}{\mathbbmss{Z}}
 \newcommand{\C}{\mathbbmss{C}}
 \newcommand{\F}{\mathbbmss{F}}
 \newcommand{\R}{\mathbbmss{R}}
 \newcommand{\Q}{\mathbbmss{Q}}



\newcommand*{\norm}[1]{\lVert #1 \rVert}
\newcommand*{\abs}[1]{| #1 |}



\newtheorem{thm}{Theorem}
\newtheorem{defn}{Definition}
\newtheorem{prop}{Proposition}
\newtheorem{lemma}{Lemma}
\newtheorem{cor}{Corollary}
\begin{document}
The aim of this page is to list examples of \PMlinkname{bounded}{BoundedOperator} and unbounded 
linear operators. 

\subsubsection*{Bounded}
\begin{itemize}
\item Identity operator, Zero operator
\item Shift operators on $\ell^p$
\item A linear operator is continuous if and only if 
   it is bounded (see \PMlinkname{this page}{ContinuousLinearMapping}).
\item Any isometry is bounded.
\item A multiplication operator $h(t) \mapsto f(t) h(t)$, where $f(t)$
  is continuous and $h\in L^p[0,1]$.
\item An integral operator $h(t) \mapsto \int_0^1 K(t,s) h(s)\,ds$, where
  $\int_0^1\int_0^1 \abs{K(s,t)}^2\,ds\,dt < \infty$ and $h\in L^2[0,1]$.
  In fact this is a Hilbert-Schmidt operator.
\item The Volterra operator $h(t) \mapsto \int_0^t h(s)\,ds$, where
  $h\in L^p[0,1]$.
\end{itemize}

\subsubsection*{Unbounded}
\begin{itemize}
\item The derivative is an unbounded operator on the 
   vector space of smooth functions equipped with the 
   $\operatorname{sup}$-norm.
\end{itemize}
%%%%%
%%%%%
\end{document}
