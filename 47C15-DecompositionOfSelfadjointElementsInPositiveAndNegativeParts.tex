\documentclass[12pt]{article}
\usepackage{pmmeta}
\pmcanonicalname{DecompositionOfSelfadjointElementsInPositiveAndNegativeParts}
\pmcreated{2013-03-22 17:51:49}
\pmmodified{2013-03-22 17:51:49}
\pmowner{asteroid}{17536}
\pmmodifier{asteroid}{17536}
\pmtitle{decomposition of self-adjoint elements in positive and negative parts}
\pmrecord{12}{40341}
\pmprivacy{1}
\pmauthor{asteroid}{17536}
\pmtype{Theorem}
\pmcomment{trigger rebuild}
\pmclassification{msc}{47C15}
\pmclassification{msc}{47B25}
\pmclassification{msc}{47A60}
\pmclassification{msc}{46L05}
%\pmkeywords{self-adjoint element decomposition}
\pmrelated{CAlgebra}

\endmetadata

% this is the default PlanetMath preamble.  as your knowledge
% of TeX increases, you will probably want to edit this, but
% it should be fine as is for beginners.

% almost certainly you want these
\usepackage{amssymb}
\usepackage{amsmath}
\usepackage{amsfonts}

% used for TeXing text within eps files
%\usepackage{psfrag}
% need this for including graphics (\includegraphics)
%\usepackage{graphicx}
% for neatly defining theorems and propositions
%\usepackage{amsthm}
% making logically defined graphics
%%%\usepackage{xypic}

% there are many more packages, add them here as you need them

% define commands here

\begin{document}
\PMlinkescapephrase{decomposition}

Every real valued function $f$ admits a well-known decomposition into its \PMlinkescapetext{positive} and \PMlinkescapetext{negative} parts: $f = f_+ - f_-$. There is an analogous result for self-adjoint elements in a \PMlinkname{$C^*$-algebra}{CAlgebra} that we will now describe.

$\,$

{\bf Theorem - } Let $\mathcal{A}$ be a $C^*$-algebra and $a \in \mathcal{A}$ a self-adjoint element. Then there are unique positive elements $a_+$ and $a_-$ in $\mathcal{A}$ such that:
\begin{itemize}
\item $a= a_+ - a_-$
\item $a_+a_- = a_-a_+ = 0$
\item Both $a_+$ and $a_-$ belong to $C^*$-subalgebra generated by $a$.
\item $\|a\| = \max\{\|a_+\|, \|a_-\|\}$
\end{itemize}

$\,$

{\bf Remark - } As a particular case, the result provides a decomposition of each self-adjoint operator $T$ on a Hilbert space as a difference of two positive operators $T=T_+ - T_-$ such that $\mathrm{Ran}\; T_- \subseteq \mathrm{Ker}\; T_+$ and $\mathrm{Ran}\; T_+ \subseteq \mathrm{Ker}\; T_-$, where $\mathrm{Ran}\;$ and $\mathrm{Ker}\;$ denote, respectively, the range and kernel of an operator.

$\,$

{\bf \emph{Proof:}}

Let us \PMlinkescapetext{fix} some notation first:
\begin{itemize}
\item $\sigma(a)$ denotes the spectrum of $a \in \mathcal{A}$.
\item $C^*[a]$ denotes the $C^*$-subalgebra generated by $a$.
\item $C_0 \big(\sigma(a)\setminus \{0\}\big)$ denotes the algebra of continuous functions in $\sigma(a)\setminus \{0\}$ that vanish at infinity.
\end{itemize}

Let $f, f_+, f_- \in C_0\big(\sigma(a)\setminus \{0\}\big)$ be the functions defined by
\begin{align*}
f(t):=t \qquad\qquad
f_+(t) :=
\begin{cases}
t, & $if$\;\; t \geq 0\\
0, & $if$\;\; t \leq 0
\end{cases}
\qquad\qquad f_-(t):=
\begin{cases}
0, & $if$\;\; t \geq 0\\
-t, & $if$\;\; t \leq 0
\end{cases}
\end{align*}
Since $a$ is \PMlinkescapetext{self-adjoint}, $\sigma(a) \subseteq \mathbb{R}$, so the above functions are well defined. It is clear that
\begin{align}
f = f_+ - f_- \;\;\;\text{and}\;\;\; f_+f_-=f_-f_+=0 \;\;\;\text{and}\;\;\; f_+, f_- \;\text{are both positive}
\end{align}


The continuous functional calculus gives an isomorphism $C^*[a] \cong C_0\big(\sigma(a)\setminus \{0\}\big)$ such that the element $a$ corresponds to the function $f$. Let $a_+$ and $a_-$ be the elements corresponding to $f_+$ and $f_-$ respectively. From the \PMlinkescapetext{observations} made in (1) it is now clear that
\begin{itemize}
\item $a_+$ and $a_-$ are both positive elements.
\item $a = a_+ - a_-$
\item $a_+a_- = a_-a_+ = 0$
\item Both $a_+$ and $a_-$ belong to $C^*[a]$.
\end{itemize}

From the fact the every $C^*$-isomorphism is isometric (see this \PMlinkname{entry}{HomomorphismsOfCAlgebrasAreContinuous}) and $\|f\| = \max\{\|f_+\|, \|f_-\|\}$ it follows that $\|a\| = \max\{\|a_+\|, \|a_-\|\}$.

The uniqueness of the decomposition follows from the uniqueness of the decomposition of real valued functions in its positive and negative parts $f = f_+-f_-$ (with $f_+f_- = 0$). $\square$
%%%%%
%%%%%
\end{document}
