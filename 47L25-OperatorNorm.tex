\documentclass[12pt]{article}
\usepackage{pmmeta}
\pmcanonicalname{OperatorNorm}
\pmcreated{2013-03-22 12:43:20}
\pmmodified{2013-03-22 12:43:20}
\pmowner{asteroid}{17536}
\pmmodifier{asteroid}{17536}
\pmtitle{operator norm}
\pmrecord{15}{33018}
\pmprivacy{1}
\pmauthor{asteroid}{17536}
\pmtype{Definition}
\pmcomment{trigger rebuild}
\pmclassification{msc}{47L25}
\pmclassification{msc}{46A32}
\pmclassification{msc}{47A30}
\pmsynonym{induced norm}{OperatorNorm}
\pmrelated{VectorNorm}
\pmrelated{OperatorTopologies}
\pmrelated{HomomorphismsOfCAlgebrasAreContinuous}
\pmrelated{CAlgebra}
\pmdefines{bounded linear map}
\pmdefines{unbounded linear map}
\pmdefines{bounded operator}
\pmdefines{unbounded operator}

% this is the default PlanetMath preamble.  as your knowledge
% of TeX increases, you will probably want to edit this, but
% it should be fine as is for beginners.

% almost certainly you want these
\usepackage{amssymb}
\usepackage{amsmath}
\usepackage{amsfonts}

% used for TeXing text within eps files
%\usepackage{psfrag}
% need this for including graphics (\includegraphics)
%\usepackage{graphicx}
% for neatly defining theorems and propositions
%\usepackage{amsthm}
% making logically defined graphics
%%%\usepackage{xypic}

% there are many more packages, add them here as you need them

% define commands here
\def\oo{\infty}
\def\V{{\mathsf V}}
\def\W{{\mathsf W}}
\def\R{{\mathbb R}}
\def\v{{\mathbf v}}
\def\op{{\rm op}}
\begin{document}
\subsection*{Definition}

Let $A\colon \V\to\W$ be a linear map between normed vector spaces $\V$ and
$\W$. To each such map (operator) $A$ we can assign a non-negative number
$\|A\|_{\op}$ defined by
\[
 \|A\|_{\op} := \mathop{\sup_{\v\in\V}}_{\v\ne{\bf 0}} \frac{\|A\v\|}{\|\v\|},
\]
where the supremum $\|A\|_{\op}$ could be finite or infinite.
Equivalently, the above definition can be written as
\[
 \|A\|_\op := \mathop{\sup_{\v\in\V}}_{\|\v\|=1} \|A\v\|
    = \mathop{\sup_{\v\in\V}}_{0<\|\v\|\le1} \|A\v\|.
\]
By convention, if $\V$ is the zero vector space, any operator from $\V$
to $\W$ must be the zero operator and is assigned zero norm.

$\|A\|_{\op}$ is called the the \emph{operator norm} (or the \emph{induced norm})
of $A$, for reasons that will be clear in the next \PMlinkescapetext{section}.

\subsection*{Operator norm is in fact a norm}

{\bf Definition -} If $\|A\|_{\op}$ is finite, we say that $A$ is a
\emph{\PMlinkescapetext{bounded}}. Otherwise, we say that $A$ is \emph{\PMlinkescapetext{unbounded}}.


It turns out that, for bounded operators, $\|\cdot\|_{\op}$ satisfies all the properties of a norm
(hence the name \emph{operator norm}). The proof follows immediately from the definition:

\begin{description}
\item[Positivity:]
  Since $\|A\v\|\ge 0$, by definition
  $\|A\|_\op \ge 0$. Also, $\|A\v\| = 0$ identically only if $A=0$.
  Hence $\|A\|_\op = 0$ only if $A = 0$.
\item[Absolute homogeneity:]
  Since $\|\lambda A\v\|=|\lambda| \|A\v\|$, by definition
  $\|\lambda A\|_\op = |\lambda| \|A\|_\op$.
\item[Triangle inequality:]
  Since $\|(A+B)\v\|=\|A\v+B\v\|\le \|A\v\| + \|B\v\|$, by definition
  $\|A+B\|_\op \le \|A\|_\op + \|B\|_\op$.
\end{description}

The set $L(\V,\W)$ of bounded linear maps from ${\mathsf V}$ to ${\mathsf W}$ forms a vector space and $\|\cdot\|_{\op}$ defines a norm in it.


\subsection*{Example}
Suppose that $\V=(\R^n,\|\cdot\|_p)$ and $\W=(\R^n,\|\cdot\|_p)$, where
$\|\cdot\|_p$ is the vector p-norm. Then the operator norm
$\|\cdot\|_\op = \|\cdot\|_p$ is the matrix p-norm.
%%%%%
%%%%%
\end{document}
