\documentclass[12pt]{article}
\usepackage{pmmeta}
\pmcanonicalname{Unitary}
\pmcreated{2013-03-22 12:02:01}
\pmmodified{2013-03-22 12:02:01}
\pmowner{asteroid}{17536}
\pmmodifier{asteroid}{17536}
\pmtitle{unitary}
\pmrecord{21}{31042}
\pmprivacy{1}
\pmauthor{asteroid}{17536}
\pmtype{Definition}
\pmcomment{trigger rebuild}
\pmclassification{msc}{47D03}
\pmclassification{msc}{47B99}
\pmclassification{msc}{47A05}
\pmclassification{msc}{46C05}
\pmclassification{msc}{15-00}
\pmsynonym{complex inner product space}{Unitary}
\pmrelated{EuclideanVectorSpace2}
\pmrelated{PauliMatrices}
\pmdefines{unitary space}
\pmdefines{unitary matrix}
\pmdefines{unitary transformation}
\pmdefines{unitary operator}
\pmdefines{unitary group}

\usepackage{amsmath}
\usepackage{amsfonts}
\usepackage{amssymb}
\newcommand{\reals}{\mathbb{R}}
\newcommand{\natnums}{\mathbb{N}}
\newcommand{\cnums}{\mathbb{C}}
\newcommand{\znums}{\mathbb{Z}}
\newcommand{\lp}{\left(}
\newcommand{\rp}{\right)}
\newcommand{\lb}{\left[}
\newcommand{\rb}{\right]}
\newcommand{\supth}{^{\text{th}}}
\newtheorem{proposition}{Proposition}
\newtheorem{definition}[proposition]{Definition}
\newcommand{\nl}[1]{\PMlinkescapetext{{#1}}}
\newcommand{\pln}[2]{\PMlinkname{{#1}}{#2}}
\begin{document}
\subsection{Definitions}
\begin{itemize}
\item A {\bf unitary space} $V$ is a complex vector space with a
distinguished positive definite Hermitian form,
$$
\langle -,-\rangle: V\times V \rightarrow \cnums,$$
which serves as
the inner product on $V$.  
\end{itemize}
\begin{itemize}
\item A {\bf unitary transformation} is a surjective linear transformation
$T:V\rightarrow V$ satisfying
\begin{equation}
  \label{eq:def}
\langle u,v \rangle = \langle Tu,Tv\rangle,\quad  u, v \in V.  
\end{equation}
These are isometries of $V$.
\end{itemize}
\begin{itemize}
\item More generally, a {\bf unitary transformation} is a surjective linear transformation $T:U \longrightarrow V$ between two unitary spaces $U,V$ satisfying
\begin{displaymath}
\langle Tv , Tu \rangle_{V} = \langle v , u \rangle_{U},\quad\; u,v \in U
\end{displaymath}
In this entry will restrict to the case of the first \PMlinkescapetext{definition}, i.e. $U = V$.
\end{itemize}
\begin{itemize}
\item A {\bf unitary matrix} is a square complex-valued matrix, $A$, whose inverse 
is equal to its conjugate transpose:
$$A^{-1}=\bar{A}^t.$$
\end{itemize}
\begin{itemize}
\item When $V$ is a Hilbert space, a bounded linear operator $T:V \longrightarrow V$ is said to be a {\bf unitary operator} if its inverse is equal to its adjoint:
\begin{displaymath}
T^{-1} = T^*
\end{displaymath}

In Hilbert spaces unitary transformations correspond precisely to unitary operators.
\end{itemize}

\subsection{Remarks}

\begin{enumerate}
\item 

A standard example of a unitary space is
$\cnums^n$ with inner product
\begin{equation}
  \label{eq:cprod}
  \langle u,v \rangle = \sum_{i=1}^n u_i\, \overline{v_i},\quad
  u,v \in\cnums^n.  
\end{equation}


\item Unitary transformations and unitary matrices
are closely related.  On the one hand, a unitary matrix defines a
unitary transformation of $\cnums^n$ relative to the inner product
\eqref{eq:cprod}.  On the other hand, the representing matrix of a
unitary transformation relative to an orthonormal basis is, in fact, a
unitary matrix.

\item A unitary transformation is an automorphism.  This follows from
  the fact that a unitary transformation $T$ preserves the
  inner-product norm:
  \begin{equation}
    \label{eq:def1}
    \Vert T u \Vert= \Vert u\Vert,\quad u\in V.    
  \end{equation}
  Hence, if $$Tu=0,$$
  then by  the definition \eqref{eq:def}
  it follows that  
  $$\Vert u \Vert = 0,$$
  and hence by the inner-product axioms that 
  $$u=0.$$
  Thus, the kernel of $T$ is trivial, and therefore it is an
  automorphism.
\item Moreover, relation \eqref{eq:def1} can be taken as the definition
  of a unitary transformation.  Indeed, using the polarization
  identity it is possible to show that if
  $T$ preserves the norm, then \eqref{eq:def} must hold as well.


\item A simple example of a unitary matrix is the change of
  coordinates matrix between two orthonormal bases.  Indeed, let
  $u_1,\ldots, u_n$ and $v_1,\ldots,v_n$ be two orthonormal bases, and
  let $A=(A^i_j)$ be the corresponding change of basis matrix
  defined by
$$v_j = \sum_i A^i_j\, u_i,\quad j=1,\ldots, n.$$
Substituting the above relation into the defining relations for an
orthonormal basis,
\begin{eqnarray*}
\langle u_i,u_j\rangle &=& \delta_{ij},\\
\langle v_k,v_l\rangle &=& \delta_{kl},  
\end{eqnarray*}
we obtain
$$\sum_{ij} \delta_{ij} A^i_k \overline{A^j_l}  = \sum_i A^i_k \overline{A^i_l} =
\delta_{kl}.$$
In matrix notation, the above is simply
$$A \bar{A}^t = I,$$
as desired.
\item Unitary transformations form a group under composition. Indeed, if $S, T$ are unitary transformations then $ST$ is also surjective and
\begin{displaymath}
\langle STu, STv \rangle = \langle Tu, Tv\rangle = \langle u, v\rangle
\end{displaymath}
for every $u,v \in V$. Hence $ST$ is also a unitary transformation.
\item Unitary spaces, transformations, matrices and operators are of fundamental
  importance in quantum mechanics.
\end{enumerate}
%%%%%
%%%%%
%%%%%
\end{document}
