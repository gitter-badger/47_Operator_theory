\documentclass[12pt]{article}
\usepackage{pmmeta}
\pmcanonicalname{IsoperimetricProblem}
\pmcreated{2013-03-22 19:12:01}
\pmmodified{2013-03-22 19:12:01}
\pmowner{pahio}{2872}
\pmmodifier{pahio}{2872}
\pmtitle{isoperimetric problem}
\pmrecord{9}{42115}
\pmprivacy{1}
\pmauthor{pahio}{2872}
\pmtype{Definition}
\pmcomment{trigger rebuild}
\pmclassification{msc}{47A60}
\pmclassification{msc}{49K22}
\pmrelated{IsoperimetricInequality}
\pmrelated{LagrangeMultiplier}

\endmetadata

% this is the default PlanetMath preamble.  as your knowledge
% of TeX increases, you will probably want to edit this, but
% it should be fine as is for beginners.

% almost certainly you want these
\usepackage{amssymb}
\usepackage{amsmath}
\usepackage{amsfonts}

% used for TeXing text within eps files
%\usepackage{psfrag}
% need this for including graphics (\includegraphics)
%\usepackage{graphicx}
% for neatly defining theorems and propositions
 \usepackage{amsthm}
% making logically defined graphics
%%%\usepackage{xypic}

% there are many more packages, add them here as you need them

% define commands here

\theoremstyle{definition}
\newtheorem*{thmplain}{Theorem}

\begin{document}
The simplest of the isoperimetric problems is the following:

\emph{One must set an arc with a given length $l$ from a given point $P$ of the plane to another given point $Q$ such that the arc together with the line segment $PQ$ encloses the greatest area possible.}

This task is solved in the entry example of calculus of variations.\\


More generally, \emph{isoperimetric problem} may \PMlinkescapetext{mean} determining such an arc $c$ between the given points $P$ and $Q$ that it gives for the integral
\begin{align}
\int_P^Q\!f(x,\,y,\,y')\,ds
\end{align}
an extremum and that gives for another integral
\begin{align}
\int_P^Q\!g(x,\,y,\,y')\,ds
\end{align}
a given value $l$, as both integrals are taken along $c$.\, Here, $f$ and $g$ are given functions.\\


The constraint (2) can be omitted by using the function $f\!-\!\lambda g$ instead of $f$ in (1) similarly as in the mentionned example. 

%%%%%
%%%%%
\end{document}
